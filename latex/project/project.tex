\documentclass[12pt]{article}
    \usepackage[includeheadfoot,a4paper]{geometry}
    \usepackage{amsfonts, amssymb, eurosym, amsmath, textcomp}
    
    \usepackage{babel} %pacote para o português
    \usepackage{cjhebrew}
    
    \usepackage{enumitem}
    
    \usepackage[utf8]{inputenc}  %acentuação
    \usepackage[pdftex]{hyperref} %link de referência
    
    \usepackage{changepage}
    
    %\usepackage{colortbl}
    %\setlength{\textwidth}{15cm}
    %\setlength{\oddsidemargin}{0.5cm}
    %\topskip=7.5mm
    %\setlength{\textheight}{26cm}
    %\setlength{\topmargin}{-2.1cm}
    
    \usepackage{calc}

\newlength\tabsize                    % Indentation
\setlength\tabsize\parindent          % Indentation
%\setlength\parindent{0pt}             % Indentation
\newcommand{\ind}{\hspace*{\tabsize}} % Indentation
\newcommand{\bnd}{\ind\ind}           % Indentation

%COUNTER TO STRING
    \newcommand{\ctrtostr}[1]{\the\roman\value{#1}}
    \newcommand{\definelr}[2]{
        \renewcommand{\leftside\ctrtostr{depth}}{\left#1}
        \renewcommand{\rightside\ctrtostr{depth}}{\right#1}
    }

%MAGIC
\newcommand{\PrmDef}[3]{
    \expandafter\newcommand\csname #1\endcsname[#2]{#3}\relax
}

%use:
%PrmDef{<FUNCTION NAME>}{<NUMBERS OF ARGUMENTS IT WILL TAKE>}{<VALUE IT WILL TAKE>}

%ENVIRONMENTS
    \newenvironment{LEMMA}[1]{{\sc Lema}:    {\em#1.}\\}{}
    \newenvironment{THEOR}[1]{{\sc Teorema}: {\em#1.}\\}{}

%TUPLE-LIKE COMMANDS
    \makeatletter
        \newcommand{\SET}{\left\{\SETnextarg}
        \newcommand{\SETnextarg}{\@ifnextchar{\bgroup}{\SETgobble}{\right\}}}
        \newcommand{\SETgobble}[1]{#1\@ifnextchar{\bgroup}{,\SETgobble}{\right\}}}

        \newcommand{\TUPLE}{\left\langle\TUPLEnextarg}
        \newcommand{\TUPLEnextarg}{\@ifnextchar{\bgroup}{\TUPLEgobble}{\right\rangle}}
        \newcommand{\TUPLEgobble}[1]{#1\@ifnextchar{\bgroup}{,\TUPLEgobble}{\right\rangle}}

        \newcommand{\POINT}{\left(\POINTnextarg}
        \newcommand{\POINTnextarg}{\@ifnextchar{\bgroup}{\POINTgobble}{\right)}}
        \newcommand{\POINTgobble}[1]{#1\@ifnextchar{\bgroup}{,\POINTgobble}{\right)}}

        \newcommand{\FMLA}{{}^\ulcorner\FMLAnextarg}
        \newcommand{\FMLAnextarg}{\@ifnextchar{\bgroup}{\FMLAgobble}{{}^\urcorner}}
        \newcommand{\FMLAgobble}[1]{#1\@ifnextchar{\bgroup}{,\FMLAgobble}{{}^\urcorner}}
    \makeatother

%ARROW-LIKE COMMANDS
    \newcommand{\rar}{\rightarrow}                          %ARROW Right
    \newcommand{\RAR}{\Rightarrow}                          %ARROW Right Meta
    \newcommand{\lar}{\leftarrow}                           %ARROW Left
    \newcommand{\LAR}{\Leftarrow}                           %ARROW Left Meta
    \newcommand{\bim}{\leftrightarrow}                      %ARROW Iff
    \newcommand{\BIM}{\Leftrightarrow}                      %ARROW Iff Meta

%MISC COMMANDS
    \newcommand{\eqc}[1]{\left\llbracket#1\right\rrbracket} %Equivalence Class
    \newcommand{\eop}{\marginpar{$\blacksquare$}}           %End of Proof
    \newcommand{\eol}{\hrule$ $\\}                          %End of line (with hr)
    \newcommand{\nl}{$ $\newline}                           %Force Newline
    \newcommand{\cls}{\newpage}                             %Shorthand for clearpage
    \newcommand*\elide{\textup{[\,\dots] }}                 %[...]
    \renewcommand{\ind}{\hspace*{\parindent}}
    \newcommand{\tripstar}{
        \begin{center}
            {\textbf{* * *}}
        \end{center}
    }
    \newcommand{\quadstar}{
        \begin{center}
            {\textbf{* * * *}}
        \end{center}
    }
    \newcommand{\ZF}{\text{\sc ZF}}
    \newcommand{\ZFC}{\text{\sc ZFC}}
    \newcommand{\mathspace}{\text{ }}
    \newcommand{\tranrel}{\mathspace\epsilon\mathspace}
    \newcommand{\strong}{\textbf}
    \newcommand{\LSig}[2]{\Sigma^{#1}_{#2}}
    \newcommand{\LPi} [2]{\Pi^{#1}_{#2}}
    \newcommand{\LDel}[2]{\Delta^{#1}_{#2}}
    \newcommand{\Min}[2]{\eqc{#1 \in #2}}
    \newcommand{\Meq}[2]{\eqc{#1  =  #2}}

%DEFAULT SETS
    \newcommand{\N}{\mathbb{N}}
    \newcommand{\Z}{\mathbb{Z}}
    \newcommand{\Q}{\mathbb{Q}}
    \newcommand{\R}{\mathbb{R}}
    \newcommand{\C}{\mathbb{C}}
    \newcommand{\V}[1]{V^{(#1)}}
    \newcommand{\M}[1]{M^{(#1)}}

%CASE DEFINITIONS
    \newcommand{\casedef}[1]{
        \begin{cases}
            \begin{align*}
                #1
            \end{align*}
        \end{cases}
    }

%ENVIRONMENTS
    \newcounter{envcount}
    \setcounter{envcount}{00}
    \newcommand{\begannewenv}{\addtocounter{envcount}{1}}

    \newenvironment{definition}[1]{\begannewenv\nl\nl
        %\textbf{\arabic{envcount}}) 
        {\sc Definição:} \textbf{#1}.
        \addcontentsline{toc}{subsection}{$\cdotp$ Def. #1}\\\ind
    }{
        \tripstar
    }

    \newenvironment{lemma}[1]{\begannewenv\nl\nl
        %\textbf{\arabic{envcount}}) 
        {\sc Lema:} \textbf{#1}.
        \addcontentsline{toc}{subsection}{$\cdotp$ Prp. #1}\\\ind
    }{}

    \newenvironment{proposition}[1]{\begannewenv\nl\nl
        %\textbf{\arabic{envcount}}) 
        {\sc Proposição:} \textbf{#1}.
        \addcontentsline{toc}{subsection}{$\cdotp$ Prp. #1}\\\ind
    }{}

    \newenvironment{theorem}[1]{\begannewenv\nl\nl
        %\textbf{\arabic{envcount}}) 
        {\sc Teorema:} \textbf{#1}.
        \addcontentsline{toc}{subsection}{$\cdotp$ Teo. #1}\\\ind
    }{}

    \newenvironment{lemma*}{\begannewenv\nl\nl
        %\textbf{\arabic{envcount}}) 
        {\sc Lema:}
    }{}

    \newenvironment{proof*}{\nl\nl\nl
        {\emph{Prova}:}\\
    }{}
    \newenvironment{proof}
        {\begin{proof*}\ind}
        {\end{proof*}}

% BIBLIOGRAPHY MESSING UP
\renewenvironment{thebibliography}{
    \chapter*{Referências Bibliográficas}
    \list{}{}
}{}

\newcommand{\bibreference}[1]{$ $\\\parbox[t]{\linegoal}{#1}\hfill}
    
    \newenvironment{keywords}{{\em Palavras-chave}: \begin{enumerate}[label=]\item$ $\small}{\end{enumerate}}
    
    \newenvironment{etapas}
    {
        \begin{enumerate}[label=]\item$ $
        \end{enumerate}
        \vspace{-1.5cm}
        \begin{enumerate}[label={\bf\arabic*${}^a$ Etapa:}, align=left]
    }
    {\end{enumerate}}
    
    \title{{\bf Modelos a Valores Booleanos}\\\em e aplicações em Teoria dos Conjuntos}
    \author{\large IME-USP - BRASIL\\\small {\em José Goudet Alvim}
    {\em Hugo Luiz Mariano (orientador)} }
    \date{2017-19}
    
    \begin{document}
        \maketitle
        \begin{keywords}
            Teoria dos Conjuntos; 
            Modelos de {\sc ZF}; 
            Provas de Consistência e Independência; 
            Álgebras Booleanas, de Heyting e seus Morfismos; 
            Lógica
        \end{keywords}
        
        \begin{abstract}
                Neste trabalho de iniciação científica, estudaremos 
                a teoria de modelos a valores booleanos de $\sc ZF$, 
                a fim de entender e produzir provas de consistência 
                e independência na teoria de conjuntos. Para tanto, 
                trataremos de álgebras de Boole; de Heyting; Filtros; 
                Morfismos; Teoria de Modelos restrita à nossa estrutura 
                e Lógica, de forma cursória.
        \end{abstract}
    
        \clearpage
        
        \section{Introdução}
        \paragraph{}
            A noção de {\em Forcing} foi introduzida por Paul Cohen em um esforço para
            provar consistência e independência, em 1963, do {\em Axioma da Escolha} e 
            da {\em Hipótese do Contínuo} sobre a Teoria de Conjuntos de Zermelo e 
            Fraenkel. 
            Desde então, ela foi reformulada e melhorada e, hoje, seus descendentes 
            teóricos são ferramentas valiosas para a teoria de conjuntos, entre outras
            áreas da matemática.
        \paragraph{}
            A Teoria de {\em Modelos a Valores Booleanos}, por sua vez, é criação 
            principalmente de Dana Scott com contribuições de Robert M. Solovay e 
            Petr Vop\v enka. Ela surge como uma opção equivalente, porém mais 
            acessível, ao {\em forcing}.
        \paragraph{}
            A seguir, sobrevoaremos esses temas, para ter uma primeira vista destes conceitos
            que aprofundaremos.
        \subsection{Breve sumário da teoria}
        \paragraph{}
            Considerando o trabalho desenvolvido sobre álgebras de Boole e Heyting; filtros e morfismos ---
            feito sob a supervisão do Prof. Hugo Luiz Mariano desde 2017 ---, temos 
            construída uma base sólida sobre a qual podemos tratar dos temas abaixo.
        \paragraph{}
            Primeiro, devemos falar sobre o que é {\em ser} modelo (de uma teoria). 
            Essencialmente, dizemos que uma estrutura é modelo de uma teoria se, de alguma forma,
            esta teoria descreve a estrutura, ou, conversamente, o domínio e as relações sobre o mesmo
            se conformam às fórmulas da teoria.
        \paragraph{}
            A ideia por trás dos modelos a valores booleanos é estabelecer um código interno à 
            teoria de conjuntos para representar objetos em um universo sintético e uma tradução 
            do nosso léxico de conjuntos para este universo inédito. Manipulações deste universo
            sintético levam a modelos com propriedades interessantes, e usamos as mesmas para 
            provar resultados sobre a teoria de conjuntos.
        \paragraph{} 
            Para entender o cerne da codificação de objetos sintéticos {\em dentro} da teoria 
            de conjuntos, e portanto, dos modelos a valores booleanos de {\sc ZF}, vamos 
            examinar um exemplo do que seria um modelo $2$-valorado:
        \paragraph{}
            Podemos nos indagar como identificar cada conjunto com um correspondente natural 
            no universo das funções $2$-valoradas. Criamos um {\em proxy} que armazena 
            informações sobre pertinência de {\em alguém} nesse Universo, isto é, é uma 
            função característica.
        \paragraph{}
            Essa identificação, no entanto, não é única: se extendermos 
            o domínio sem alterar o suporte, não estamos codificando um {\em outro} objeto. 
            Além disso, ela não é homogênea: o seu domínio não está na classe que queremos 
            {\em tornar} universo.
        \paragraph{}
            Queremos, também, que valham as leis de Leibniz.
            Mas, sob a ótica da igualdade ``$=$'' de {\sc ZF}, duas 
            representações do mesmo conjunto podem ser diferentes 
            (uma ser extenção da outra, por exemplo). Queremos fundar uma estrutura 
            que, sob uma intepretação, concorda com a Teoria dos 
            conjuntos, para, portanto, ser um modelo.
        \paragraph{}
            Para tanto, criaremos uma relações de igualdade e pertinência 
            diferentes na classe das funções homogêneas $2$-valoradas, que 
            é a seguinte classe:\\
        Para cada $\alpha$ ordinal:
            $$V^{(2)}_\alpha=\{x:Func(x)\land Im(x)\subseteq 2\land\exists\beta<\alpha:Dom(x)\subseteq V^{(2)}_{\beta}\},$$
            $$V^{(2)}=\{x\in V:\exists\alpha: x\in V^{(2)}_\alpha\}.$$
        A definição da relação de igualdade e de ${}\in{}$ é mais complicada, então não será desenvolvida aqui. Mas, dispondo delas,
        $\tuple{V^{(2)}}{\widetilde=}{\widetilde\in}\vDash{\sc ZF}$.
        \paragraph{}
            Este é um exemplo de um modelo a valores booleanos. A ideia geral é repetir 
            este processo de recursão transfinita sobre uma algebra booleana completa 
            $B$ qualquer no lugar do $2$, gerando uma classe $V^{(B)}$ de funções 
            homogêneas B-valoradas, que vai carregar algum significado para as 
            fórmulas da teoria de $ZF$, particularmente, nos interessa saber 
            o que pensa $V^{(B)}$ sobre uma sentença, isto é, se 
            $$\left\langle V^{(B)},\widetilde=,\widetilde\in\right\rangle\vDash\sigma.$$
            Ou, alternativamente, se $V^{(B)}$ acredita que $\sigma$ vale.
        \paragraph{}
            A importância deste tipo de ferramenta é o fato de que, selecionando $B$ 
            com cuidado, pode-se {\em manipular} nosso $V^{(B)}$ e suas 
            $B$-verdades. Veremos que, para cada $B$ algebra booleana completa, $V^{(B)}$ 
            é modelo de ZF, isto é, os teoremas de ZF valem todos em $V^{(B)}$. Veremos, 
            também, que --- a depender de $B$ e $B'$ --- um número de coisas são verdade em 
            $V^{(B)}$ e falsas em $V^{(B')}$, isto é: estes universos discordam em algumas verdades
            fundamentais. Por exemplo, pode ser o caso de $V^{(B)}\vDash\mathfrak{c}=\aleph_1$ e
            $V^{(B')}\vDash\mathfrak{c}=\aleph_2$. Que, no caso, equivale a provarmos a independência
            da hipótese do contínuo.
    
        \section{Objetivos e Justificativa}
            \paragraph{}
                Provas de consistência e independência não são cobertas na graduação,
                e explorar um dos mecânismos teóricos que nos fornece estratégias para alcança-las
                é interessante tanto como motivador de estudo de teorias pré-requisitas --- 
                conjuntos, lógica, modelos, álgebras e reticulados ---, bem como um fim em si.
            \paragraph{} 
                Esse tipo de estudo visa aproximar um aluno de graduação ao trabalho matemático 
                em seu ambiente ``natural'': A pesquisa.
            
        \section{Plano de Trabalho e Cronograma}
            \begin{etapas}
                \item {\sc Teoria básica e preliminares} (já finalizada):
                \begin{enumerate}[label=$*$]
                    \item Posets e Reticulados.
                    \item Álgebras de Heyting e Boole.
                    \item Morfismos, Congruências e Filtros.
                \end{enumerate}
                \cite{Bell}, \cite{Jech}, \cite{Miraglia}.
                
                \item {\sc Modelos de teoria de conjuntos}:       
                \begin{enumerate}[label=$*$]
                    \item Intrudução à teoria de modelos para teoria de conjuntos.
                    \item Exemplos (Modelos transitivos).
                    \item Modelos a valores booleanos (definição e desenvolvimento).
                \end{enumerate}
                \cite{Bell}, \cite{Drake}, \cite{Jech}, \cite{Kunen}.
                
                \item {\sc Aplicações em teoria de conjuntos}:    
                \begin{enumerate}[label=$*$]
                    \item Validação.
                    \item Consistência.
                    \item Independência.
                \end{enumerate}
                \cite{Bell}, \cite{Jech}. (outros?)
    
                \item {\sc Relações fora da teoria} (se houver tempo para elas):
                \begin{enumerate}[label=$*$]
                    \item Relação com {\em forcing}.
                    \item $\Omega$-{\sc sets}.
                    \item Relação com validação em Categorias.
                \end{enumerate}
                \cite{Bell}, \cite{Jech}. (outros?)
            \end{etapas}
    
        \begin{thebibliography}{MMMM}
    
            \bibitem [{\bf Bell}] {Bell} Bell, J. L. {\bf Set Theory, Boolean-Valued Models and Independence Proofs},
                Oxford logic Guides {\bf v. 47}, Clarendon Press, 2005.
    
            \bibitem [{\bf Drake}] {Drake} Drake, Frank R. {\bf Set theory : an introduction to large cardinals}, 
            Studies in logic and the foundations of mathematics {\bf v. 76}, American Elsevier Pub. Co., 1974.
    
            \bibitem [{\bf Jech}] {Jech} Jech, T. {\bf Set Theory: Third Millennium Edition, revised. and extended}, 
                Springer Monographs in Mathematics, Springer-Verlag, 2003.
    
            \bibitem [{\bf Kunen}] {Kunen} Kunen K. {\bf Set Theory}, Studies in Logic {\bf v. 34}, 
                Lightning Source, Milton Keyne, UK.
    
    
            \bibitem [{\bf FMN}] {Miraglia} Neto, F. Miraglia. {\bf Cálculo Proposicional: Uma interação da Álgebra e da Lógica},
                Coleção CLE -- {\bf v. 1}, Centro de Lógica, Epistemologia e História da Ciência, Campinas --- São Paulo, 1987. 
        
        \end{thebibliography}
    \end{document}
    
    %Models, logics, and higher-dimensional categories : a tribute to the work of Mih aly Makkai
    %The foundations of mathematics / Kenneth Kunen.
    %Marker, D	Model theory : an introduction.
    %Hodges, Wilfrid	Model theory.
    %Chang, Chen Chu	Model theory.
    %Shelah, Saharon	Proper forcing.
    %Set theory and model theory : proceedings.