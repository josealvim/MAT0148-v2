\chapter{Teoria de Modelos de Teorias de Conjuntos}
    \cls
    \paragraph{}
        A teoria dos modelos das teorias de conjuntos está relacionada 
        com, por um lado, estudo de grandes cardinais e seus ramos, e,
        por outro, lógica e teoria de conjuntos em si.
    \paragraph{}
        Isto pois, para uma certa classe de teorias de primeira ordem 
        expressivas o suficiente, existem sentenças que são independentes
        da teoria, e é o caso que as teorias de conjuntos estão justamente 
        entre incompletas.
    \paragraph{}
        No campo da teoria, isto nos diz que a uma teoria de conjuntos é,
        de certa forma, agnóstica a respeito de certas proposições. Por 
        exemplo, é necessário que não se possa provar a existência de 
        cardinais excessivamente grandes, isto porque se um cardinal for 
        $\beth$-fixo, a hierarquia cumulativa até o mesmo é um modelo 
        da teoria de conjuntos.
    \paragraph{}
        Dado um modelo de, digamos $\ZF$ pode muito bem ser o caso que 
        haja cardinais inacessíveis no mesmo, mas sem informações 
        adicionais, é impossível provar que existem --- se a teoria for
        consistente, que esperamos ser ---.
    \paragraph{}
        No campo prático temos ainda traços do abalo deste quarto golpe
        narcísico que tomou a humanidade com os resultados de K. Gödel,
        da mesma forma que em um dado modelo de $\ZF$ possa valer ou 
        não valer $\varphi$, pode ser que problemas importantes ou 
        interessante sejam, simplesmente, independentes da teoria sem que 
        saibamos. É este justamente o caso da hipótese do contínuo, que 
        novamente mostra a capacidade de $\R$ de apresentar-se como a besta
        que de fato é.
    \paragraph{}
        A hipótese do contínuo não é \emph{um pouco} independente, por sinal.
        Como enunciou Robert M. Solovay: 
        ``\emph{$2^{\aleph_0}$ can be anything it ought to be}''\footnote{
            The Theory of Models, Proceedings of the 1963 International Symposium at Berkeley. Amsterdam, North-Holland, 1965, Addison, Henkin, Tarski, eds., pg. 435.
        }. Dizendo que $\mathfrak{c}$ pode ser (e é em algum modelo) 
        $\aleph_\beta$ para um $\beta$ sucessor ou de cofinalidade incontável.
        A hipótese do contínuo é ``tão'' independente que $\mathfrak{c}$ pode,
        inclusive, ser fracamente inacessível. 
    \paragraph{}
        Provar a independência de uma proposição pode ser feita tanto 
        sintaticamente, ou semanticamente. Enquanto a maneira sintática 
        é mais econômica ontologicamente falando, a semantica é mais 
        acessível à mente, por mais que devamos tomar cuidado para não 
        cairmos em confusões linguísticas. 
    \paragraph{}
        Empregaremos uma abordagem principalmente semantica para tratar 
        de $\ZF$, e por isso que surge a necessidade da teoria dos modelos.
        Porém, para tratar de $\ZF$, vamos primeiro definir qual teoria de 
        fato falamos.
    \paragraph{}
        A língua de nossa teoria é relativamente simples, é a lingua com 
        dois\footnote{
            Poderiamos fazer apenas com ${}\in$, mas não é necessário.
        } símbolos relacionais $\TUPLE{{}={}}{{}\in{}}$ apenas. Já a 
        teoria é a gerada por esses 7 axiomas e o Esquema de Substituição, 
        que nos dá um axioma para cada fórmula conforme.
    \begin{enumerate}[label=\arabic*. Ax.]
        \item da Identidade:\\
            $\forall x: [\exists! y: x=y] \land x=x$.
        \item Extensionalidade:\\
            $\forall x:\forall y:[\forall z: z\in x \bim z \in y]\bim x = y$.
        \item da União:\\
            $\forall x:\exists y: \forall z: [z\in y\bim\exists w\in x: z\in w]$.
        \item da Potência:\\
            $\forall x:\exists y: \forall z: z\in y \bim [\forall w: w\in z\bim w\in y]$.
        \item Esquema da Substituição:\\
            Se $\varphi$ uma fórmula com apenas $a,b,\vec{v}$ livres e $c,w,x,y,z$ não ocorrendo 
            em $\varphi$. Então:
            \begin{align*}
                [\forall\vec{v}:\forall a:[\exists b:\varphi(a,b;\vec{v})]\bim&[\exists!b:\varphi(a,b;\vec{v})]]\rar\\
                \rar&[\forall x:\exists y: \forall z: z\in y \bim \exists w: w\in x\land \varphi(w,z;\vec{v})]
            \end{align*}
        \item do Conjunto Indutivo:
        \begin{align*}
            \exists I:\exists x: x\in I \land&(\forall t: t\in x   \bim t\not=t)\\
                                        \land&[(\exists w: w\in I) \rar \exists z:z\in I \land \forall t': t'\in z \bim t'=w\lor t'\in w].
        \end{align*}
        \item da Fundação:\\
            $\forall x:(\exists x': x'\in x)\rar\exists y: (y\in x)\land[\forall t: (t\in x\land t\in y)\rar t\not=t]$.
    \end{enumerate}
    \section{Estruturas Transitivas}
        \begin{definition}{Identidade Induzida}
            Se tivermos uma relação $R$ definida sobre uma 
            classe $A$, gostaríamos de ter uma relação de 
            equivalência em $A$ que fosse congruente com 
            $R$, definimos $\approx_R$ como sendo
            $$ a \approx_R b \BIM \forall t\in A: tRa\bim tRb$$
            \paragraph{}
                Restrito ao domínio adequado.
        \paragraph{}
            Isto é, em $A$, dois identificados são idênticos à 
            esquerda.
        \end{definition}
        \begin{definition}{Estrutura Transitiva}
            Uma estrutura $\mathfrak{A} = \TUPLE{A}{\approx}{\epsilon{}}$
            é dita uma \strong{estrutura transitiva} exatamente quando 
            $\epsilon = \SET{\TUPLE{a,b}: a,b\in A\land a\in b}$, $\approx{}\!\!\!=\!\!\!{}\approx_\epsilon$ 
            e $A$ é uma classe transitiva.
        \end{definition}
        \begin{definition}{Substrutura}
            Dadas duas estruturas compatíveis
            $\mathfrak{A}$ e $\mathfrak{B}$ 
            dizemos que $\mathfrak{A}\subseteq\mathfrak{B}$ --- que $\mathfrak{A}$ 
            é substrutura de $\mathfrak{B}$ --- exatamente quando 
            $|\mathfrak{A}|\subseteq|\mathfrak{B}|$, as relações e funções de 
            $\mathfrak{A}$ são as restrições das de $\mathfrak{B}$ e as 
            constantes de $\mathfrak{B}$ são as mesmas que as de $\mathfrak{A}$.
        \end{definition}
        \begin{theorem}{Estrutura transitivas atendem fundação e extensionalidade}
            Se $\mathfrak{A} = \TUPLE{A}{\approx}{\epsilon}$ for uma estrutura transitiva, então,
            \begin{enumerate}[label=\alph*)]
                \item $\mathfrak{A}\vDash\text{Ax. da Fundação}$
                \item $\mathfrak{A}\vDash\text{Ax. da Extensionalidade}$
            \end{enumerate}
            \begin{proof}
                \begin{enumerate}[label=\alph*)]
                    \item
                        Seja $x\in A$, com tal que $\exists y\in A: y\tranrel x$, 
                        isso significa que $\exists y: y\in x$, assim, pelo axioma da fundação, 
                        $\exists y\in x:\forall t: [t\in x\land t\in y]\rar t\not = t$. Como $A$ é 
                        transitivo, temos que este $y$ existe \emph{em $A$}, então 
                        $\exists y\tranrel x:\forall t: [t\tranrel x\land t\tranrel y]\rar t\not = t$.
                    \paragraph{}
                        Por outro lado, temos que $\forall a,b\in A:a\approx b \bim a = b$, pois, para
                        a ida\footnote{a volta é trivial}: Como $A$ é transitivo, então $a,b\subset A$. Assim, se os membros de $a$ 
                        em $A$ forem exatamente os de $b$ em $A$, então os membros de $a$ são os mesmos
                        que os de $b$, e por extensionalidade, são iguais. Assim, temos
                        $\exists y\tranrel x:\forall t: [t\tranrel x\land t\tranrel y]\rar t\not \approx t$,
                        que é a tradução da fundação para a estrutura $\mathfrak{A}$.
                    \item
                        Se $a,b\in A$ então todos os membros \emph{deste} estão em $A$ também. Se os membros 
                        de $a$ e $b$ que estão dentro de $A$ coincidem, então os fora de $A$ coincidem e 
                        temos a extensionalidade. Assim, eles são iguais ($=$), mas se são iguais, como vimos,
                        também são iguais ($\approx$). Vale então a extensionalidade.
                \end{enumerate}
            \end{proof}\eop
        \end{theorem}
        \paragraph{}
            Neste teorema aparece insinuada uma propriedade de certas fórmulas que 
            chamamos de Incondicionalidade, ou \emph{Absoluteness}.
        \subsection{Incondicionalidade e Elementaridade}
            \begin{definition}{Incondicionalidade}
                    Sejam $\mathfrak{A}\subseteq\mathfrak{B}$ estruturas transitivas.
                \paragraph{}
                    Uma fórmula $\varphi$ da língua de $\ZF$ é dita \strong{absoluta} ou
                    \strong{incondicional} entre $\mathfrak{A}$ e $\mathfrak{B}$} exatamente 
                    quando, para toda $f$ interpretação das variáveis da língua em $\mathfrak{A}$.
                    $$ \mathfrak{A}\vDash_f \varphi \BIM \mathfrak{B}\vDash_f \varphi $$
                \paragraph{}
                    Um termo $t$ da língua é dito \strong{absoluta entre 
                    $\mathfrak{A}$ e $\mathfrak{B}$} exatamente quando, para toda 
                    $f$ interpretação das variáveis da língua em $\mathfrak{A}$.
                    $$ \mathfrak{A}\vDash_f x = t \BIM \mathfrak{B}\vDash_f x = t $$
                \paragraph{}
                    Dizemos ainda que uma fórmula é preservada sob restrição de $\mathfrak{B}$
                    para $\mathfrak{A}$ quando a implicação da direita para esquerda vale. E 
                    Dizemos que uma fórmula é preservada sob extensão de $\mathfrak{A}$ para 
                    $\mathfrak{B}$ quando a implicação da esquerda para a direita vale.
            \end{definition}
            \begin{definition}{Elementaridade}
                Dadas $\mathfrak{A}\subseteq\mathfrak{B}$ estruturas compatíveis
                Dizemos $\mathfrak{A}$ ser substrutura elementar de $\mathfrak{B}$,
                ou que $\mathfrak{B}$ é extensão elementar de $\mathfrak{A}$, 
                exatamente quando toda fórmula é absoluta entre $\mathfrak{A}$ e $\mathfrak{B}$.
            \end{definition}
            \paragraph{}
                Fórmulas absolutas, pois, formam uma classe de fórmulas muito útil para o trato
                de modelos, já que seus significados não mudam quando as extendemos ou as
                restringimos. Que estas fórmulas não são todas as que existem é simple de ver:
                Deixe
                $\mathfrak{A} = \TUPLE{\SET{1}{2}}{\leq}$  e
                $\mathfrak{B} = \TUPLE{\SET{1}{2}{3}}{\leq}$
                É trivial ver que $\forall x,y,z: (x\not= y)\rar z=x\lor z=y$ 
                não é absoluta entre as estruturas.
            \paragraph{}
                No entanto, a situação não é tão ruim assim, por mais que tamanho não 
                seja absoluto entre estruturas, temos critérios para gerar fórmulas 
                absolutas:
            \begin{proposition}{Condições suficientes para incondicionalidade}
                \begin{enumerate}[label=\alph*)]
                    \item fórmulas atômicas são absolutas.
                    \item conjunção, disjunção e negação de absolutas é a absoluta.
                \end{enumerate}
                \begin{proof}
                    \begin{enumerate}[label=\alph*)]
                        \item 
                            Como uma é substrutura da outra, então as relações são as 
                            restrições. Como a fórmula é atômica, e a interpretação é 
                            em na substrutura, então vai ser verdade em numa estrutura
                            exatamente quando for na outra.
                        \item 
                            Segue da definição de satisfação.
                    \end{enumerate}
                \end{proof}
            \end{proposition}
            \paragraph{}
            \begin{definition}{Fórmulas Completas}
                    Uma fórmula $\varphi(x;\vec{v})$ é dita \strong{completa} 
                    em $\mathfrak A\subseteq\mathfrak{B}$ com respeito a $x$ 
                    exatamente quando é o caso que:
                \paragraph{}
                    Se $\hat{x}\in|\mathfrak{B}|$ e $\vec{u}$ forem parâmetros 
                    em $|\mathfrak{A}|$, então $\mathfrak{B}\vDash\varphi(\hat x, 
                    \vec u) \RAR b\in|\mathfrak{A}|$. Sendo que com ``
                    $\mathfrak{B}\vDash\varphi(\hat x, \vec u)$'' queremos dizer
                    ``$\mathfrak{B}\vDash_{f_{[x\slash\hat x, \vec v\slash\vec u]}}\varphi(\hat x, \vec u)$''
                    pois $\varphi$ só tem os parâmetros e $x$ livres.
                \paragraph{}
                    Uma fórmula completa em relação a uma par estrutura-substrutura 
                    e um variável é de tal forma que se não for verdade em baixo,
                    não é por falta de testemunha. Da mesma forma que se uma sequência 
                    em $\R$ não converge, não é porque está faltando o ponto de 
                    convergência, como poderia ser o caso em $\Q$. 
            \end{definition}
            \begin{theorem}{$\exists x:\varphi$ para completas e absolutas}
                Seja $\varphi(x;\vec v)$ absoluta entre $\mathfrak{A}\subseteq\mathfrak{B}$ transitivas, e 
                completa para $x$ entre as mesmas estruturas. Nestas condições, 
                $\exists x:\varphi(x,\vec v)$ é absoluta entre $\mathfrak{A}$ e $\mathfrak{B}$.
                \begin{proof}
                    \paragraph{}
                        Primeiro, temos que:
                    \begin{enumerate}[label=\alph*)]
                        \item $\mathfrak{A}\vDash_f\varphi(x;\vec v) \BIM \mathfrak{B}\vDash_f\varphi(x;\vec v)$.
                        \item $[(\vec u \subseteq|\mathfrak{A}|)\land(b\in|\mathfrak{B}|)]\RAR[\mathfrak{B}\vDash_f\varphi(b;\vec u) \RAR b\in|\mathfrak{A}]$.
                    \end{enumerate}
                    \paragraph{($\RAR$)}
                        Então considere:
                        \begin{prooftree}
                            \AxiomC{$\mathfrak{A}\vDash_f\exists x:\varphi(x,\vec v)$}
                            \UnaryInfC{Existe um $\hat x$ em $|\mathfrak{A}|$ tal que: $\mathfrak{A}\vDash_{f_{[x\slash{\hat x}]}}\varphi(x,\vec v)$}
                            \UnaryInfC{Existe um $\hat x$ em $|\mathfrak{B}|$ tal que: $\mathfrak{A}\vDash_{f_{[x\slash{\hat x}]}}\varphi(x,\vec v)$}
                            \UnaryInfC{Existe um $\hat x$ em $|\mathfrak{B}|$ tal que: $\mathfrak{B}\vDash_{f_{[x\slash{\hat x}]}}\varphi(x,\vec v)$}
                            \UnaryInfC{$\mathfrak{B}\vDash_f\exists x:\varphi(x,\vec v)$}
                        \end{prooftree}
                    \paragraph{($\LAR$)}
                    Por outro lado, tome que
                        \begin{prooftree}
                            \AxiomC{$\mathfrak{B}\vDash_f\exists x:\varphi(x,\vec v)$}
                            \UnaryInfC{Existe um $\hat x$ em $|\mathfrak{B}|$ tal que: $\mathfrak{B}\vDash_{f_{[x\slash{\hat x}]}}\varphi(x,\vec v)$}
                            \AxiomC{$f$ é uma $\mathfrak{A}$-valoração de variáveis}
                            \UnaryInfC{$f[\vec v]\subseteq|\mathfrak{A}|$}
                            \BinaryInfC{Existe um $\hat x$ em $|\mathfrak{A}|$ tal que: $\mathfrak{A}\vDash_{f_{[x\slash{\hat x}]}}\varphi(x,\vec v)$}
                            \UnaryInfC{$\mathfrak{A}\vDash_f\exists x:\varphi(x,\vec v)$}
                        \end{prooftree}
                    \paragraph{}
                        Assim, fica provado o teorema.\eop
                \end{proof}
            \end{theorem}
            \begin{theorem}{Quantificação limitada de fórmulas absolutas}
                Seja $\varphi$ absoluta entre $\mathfrak{A}\subseteq\mathfrak{B}$ estruturas transitivas, então $\exists x\in y: \varphi(x)$ e $\forall x\in y: \varphi(x)$
                são ambas absolutas, quando $y$ não ocorre em $x$.
                \begin{proof}
                    Primeiro, $\exists x\in y: \varphi(x) \equiv \exists x:x\in y\land \varphi$:
                    \begin{prooftree}
                        \AxiomC{$x\in y$ é absoluta}
                        \AxiomC{$\varphi(x)$ é absoluta}
                        \BinaryInfC{$x\in y\land\varphi(x)$ é absoluta}
                    \end{prooftree}
                    \paragraph{}
                        $x\in y\land\varphi(x)$ é completa para $x$ entre $\mathfrak{A}$ e $\mathfrak{B}$ pois:
                        se $y\in\mathfrak{A}$ e $\mathfrak{B}$ crê que $x$ está em $y$ e satisfaz $\varphi$, 
                        então, certamente $x$ está em $y$ que está em $\mathfrak{A}$.
                    \paragraph{}
                        Logo, pelo teorema anterior, $\exists x\in y: \varphi(x)$ é absoluta entre as estruturas.
                    \paragraph{}
                        Para o quantificador universal, basta ver que $\neg\varphi(x)$ também é absoluta, então 
                        $\exists x: x\in y\land \neg \varphi(x)$ é absoluta, mas ela é equivalente a $\exists x:
                        \neg(x\in y \rar \varphi(x))$ que equivale a $\neg\forall x: x\in y\rar\varphi(x)$. 
                        Neste caso, sabemos que ela será absoluta, mas negação de absoluta também é. Então 
                        $\forall x: x\in y\rar\varphi(x)$.\eop
                \end{proof}
            \end{theorem}
            \begin{definition}{Fórmulas preservadas por Extenção/Restrição}
                \paragraph{}
                    Dizemos que uma fórmula $\varphi$ é \strong{preservada por extenções} --- ou é \strong{extendível} ---
                    exatamente quando:
                \begin{center}
                    $\mathfrak{A}\subseteq\mathfrak{B}$ estruturas transitivas $\RAR ( \mathfrak{A}\vDash_f\varphi\RAR\mathfrak{B}\vDash_f\varphi)$
                \end{center}
                Toda vez que $f$ for uma $\mathfrak{A}$ valoração.
                \paragraph{}
                    Dizemos que uma fórmula $\varphi$ é \strong{preservada por restrições} --- ou é \strong{restringível} ---
                    exatamente quando:
                \begin{center}
                    $\mathfrak{A}\subseteq\mathfrak{B}$ estruturas transitivas $\RAR ( \mathfrak{A}\vDash_f\varphi\LAR\mathfrak{B}\vDash_f\varphi)$
                \end{center}
                Toda vez que $f$ for uma $\mathfrak{A}$ valoração.
            \end{definition}
            \begin{lemma}{Quantificação sobre absoluta é Restringível/Extendível}
                Trivialmente, se $\varphi$ for absoluta entre duas transitivas $\mathfrak{A}\subseteq\mathfrak{B}$, então $\exists x:\varphi$ é extendível e 
                $\forall x:\varphi$ é restringível.
            \end{lemma}
        \subsection{Hierarquia de Lévy}
            \begin{definition}{Fórmulas $\Sigma$, $\Pi$ e $\Delta$}
                Uma fórmula $\varphi$ é dita \strong{restrita} ou \strong{limitada} quando todos os 
                seus quantificadores são da forma $\forall x : x\in y \rar \psi$ ou 
                $\exists x : x\in y \land \psi$
                (isto é $\forall x\in y : \psi$ ou $\exists x\in y : \psi$ )
            \paragraph{}
                Uma fórmula é dita $\Sigma_0$ e $\Pi_0$ exatamente quando ela é restrita; É dita
                $\Sigma_{n+1}$ quando é da forma $\exists x_1:\ldots:\exists x_k: \psi$ para 
                $(x_i)_{1\leq i\leq k}$ variáveis e uma 
                $\psi$ fórmula $\Pi_n$. Similarmente, é dita $\Pi_{n+1}$ exatamente quando 
                $\forall x_1:\ldots:\forall x_k\psi$ com $(x_i)_{1\leq i\leq k}$ variáveis 
                e uma $\psi$ fórmula $\Sigma_n$.
            \paragraph{}
                Finalmente, uma fórmula $\varphi$ é dita $\LSig{\theoryof{}}{n}$ --- sigma-$n$ para 
                $\theoryof{}$ --- quando existe uma $\psi$ que é $\LSig{}{n}$ tal que $\theoryof{}\vdash\psi\bim\varphi$,
                $\LPi{\theoryof{}}{n}$ sendo similarmente definida.
                Um caso especial são as fórumlas $\LDel{\theoryof{}}{n}$, que são exatamente aquelas 
                fórmulas que são $\Sigma^{\theoryof{}}_n$ e $\Pi^{\theoryof{}}_n$ simultaneamente. 
                Novamente, um termo $t$
                é dito $\LSig{\theoryof{}}{n}$, $\LPi{\theoryof{}}{n}$ ou $\LDel{\theoryof{}}{n}$ 
                quando $x = t$ o for --- com $x$ não ocorrendo em $t$, é claro ---.
            \end{definition}
            \paragraph{}
                O motivo do nosso interesse em catalogar certas fórmulas na hierarquia de Lévy é a relação que 
                estas fórmulas possuem com as estruturas transitivas. A ver, as fórmulas 
                $\LSig{\theoryof{}}{1}$ são preservadas por extensões, as $\LPi{\theoryof{}}{1}$ são 
                preservada por restrições e as $\LDel{\theoryof{}}{1}$ são absolutas, quando se tratando de 
                modelos da teoria $\theoryof{}$, claro.
            \begin{theorem}{Fórmulas $\LDel{\theoryof{}}{1}$ são $\theoryof{}$-absolutas}
                \begin{proof}
                    \begin{lemma}{$\LDel{}{0}$ são absolutas}
                        Trivialmente, pois são conjunções de outras absolutas ou quantificações limitadas de 
                        fórmulas absolutas. 
                    \end{lemma}
                    \paragraph{}
                        Seja $\varphi$ uma fórmula $\LDel{\theoryof{}}{1}$. Como ela é equivalente a uma $\forall x_k:\psi$ com 
                        $\psi\in\LSig{\theoryof{}}{0}$, ela é $\theoryof{}$-equivalente a uma quantificação universal de uma 
                        fórmula absoluta, afinal $\psi$ é absoluta.
                    \paragraph{}
                        Por outro lado, ela é equivalente a uma $\exists x_j:\gamma$ com $\gamma\in\LPi{\theoryof{}}{0}$, que 
                        nos dá que ela é $\theoryof{}$-equivalente a uma quantificação existêncial de uma absoluta, pois $\gamma$
                        também é absoluta.
                    \paragraph{}
                        Assim, sendo $\mathfrak{A}\subseteq\mathfrak{B}$ estruturas transitivas e $f$ uma $\mathfrak{A}$-valoração, 
                        temos que:
                    \begin{enumerate}[label=\alph*)]
                        \item $\theoryof{}\vdash\varphi\bim\forall x_k:\psi$.
                        \item $\theoryof{}\vdash\varphi\bim\exists x_k:\gamma$.
                        \item $\mathfrak{A}\vDash_f\exists x_k:\gamma\RAR\mathfrak{B}\vDash_f\exists x_k:\gamma$.
                        \item $\mathfrak{A}\vDash_f\exists x_k:\psi  \RAR\mathfrak{B}\vDash_f\forall x_k:\psi$.
                    \end{enumerate}
                    \paragraph{}
                        Se for o caso que $\mathfrak{A},\mathfrak{B}\vDash\theoryof{}$, então temos que
                        $$\mathfrak{A}\vDash_f\varphi\BIM\mathfrak{B}\vDash_f\varphi$$\eop
                \end{proof}
            \end{theorem}
            \paragraph{}
                Agora que temos uma condição suficiente para uma fórmula ser absoluta entre estruturas 
                transitivas de uma teoria dada, transferimos o problema de identificar uma fórmula 
                absoluta para o problema de identificar uma fórmula $\LDel{\theoryof{}}{1}$
            \paragraph{}
                Por \cite{DRAKE}, temos que, para um teoria $\theoryof{}$ tão forte quanto $\ZF$,
                \begin{theorem}{Resultados acerca da Hierarquia de Lévy}
                    \begin{enumerate}[label=(\alph*)]
                        \item Se $\text{``}\varphi\text{''}\in\LSig{\theoryof{}}{n}$, então $\text{``}\forall x:\varphi\text{''} \in\LPi{\theoryof{}}{n+1}$ e 
                            $\text{``}\exists x:\varphi\text{''}\in\LSig{\theoryof{}}{n}$, para toda $x$ variável da língua.

                        \item Se $\text{``}\varphi\text{''}\in\LPi{\theoryof{}}{n}$,  então $\text{``}\forall x:\varphi\text{''} \in\LPi{\theoryof{}}{n}$ e 
                            $\text{``}\exists x:\varphi\text{''}\in\LSig{\theoryof{}}{n+1}$, para toda $x$ variável da língua.
                            
                        \item Se $\text{``}\varphi\text{''}, \text{``}\psi\text{''}\in\LPi{\theoryof{}}{n}$, então  suas conjunções e disjunções são $\LPi{\theoryof{}}{n}$

                        \item Se $\text{``}\varphi\text{''}, \text{``}\psi\text{''}\in\LSig{\theoryof{}}{n}$, então suas conjunções e disjunções são $\LSig{\theoryof{}}{n}$

                        \item Se $\text{``}\varphi\text{''}\in\LSig{\theoryof{}}{n}$ e $\text{``}\psi\text{''}\in\LPi{\theoryof{}}{n}$, então suas conjunções e disjunções
                            são $\LDel{\theoryof{}}{n+1}$,\\ 
                            e $\text{``}\varphi\rar\psi\text{''}$ é $\LPi{\theoryof{}}{n}$\\
                            e $\text{``}\psi\rar\varphi\text{''}$ é $\LSig{\theoryof{}}{n}$
                        \item Quantificação limitada não altera classe de Lévy: portanto\\
                                Se $\text{``}\varphi\text{''}\in\LSig{\theoryof{}}{n}$, temos que ``$\forall x\in y:\varphi$'' continua $\LSig{\theoryof{}}{n}$,
                        \item Se $\varphi(x)$ for         $\LDel{\theoryof{}}{n}$,  então $\SET{x:\varphi(x)}$ é $\LPi{\theoryof{}}{n}$
                        \item Se $\varphi(x)$ for         $\LDel{\theoryof{}}{n}$,  então $\SET{x\in y:\varphi(x)}$ é $\LDel{\theoryof{}}{n}$
                        \item Se $x$ for um termo         $\LSig{\theoryof{}}{n}$   e $\theoryof{}\vdash\exists y: y = x$, então $t$ é $\LDel{\theoryof{}}{n}$
                        \item Se $\varphi(x)$ e $t$ forem $\LDel{\theoryof{}}{n}$   e $\theoryof{}\vdash\exists y: y = t$, então todos são $\LDel{\theoryof{}}{n}$:\\
                                $\bullet$ $\SET{x\in y:\varphi(x)}$ é $\LDel{\theoryof{}}{n}$   \\
                                $\bullet$ $\exists x\in t: \varphi(x)$                          \\
                                $\bullet$ $\forall x\in t: \varphi(x)$                          
                        \item Se $\varphi$ e $t$ forem $\LSig{\theoryof{}}{n}$, então $\SET{t:\varphi}$ é $\LDel{\theoryof{}}{n+1}$
                        \end{enumerate}
%                    \begin{lemma}{Algumas Fórmulas e Termos $\LDel{\ZF}{0}$ ou $\LDel{\ZF}{1}$}
%                        {
%
%                        $$\LDel{\ZF}{0}:$$ 
%                        $
%                            x = y           ,
%                            x \subseteq y   ,
%                            \emptyset       ,
%                            \SET{x}{y}      ,
%                            \TUPLE{x}{y}    .
%                        $
%                        
%                        } {
%                        
%                        $$\LDel{\ZF}{1}:$$ $   
%                            Trans(x)        ,
%                            Ord(x)          ,
%                            x<y(x)          ,
%                            \omega          ,
%                            Fun(f)          ,
%                            Dom(f)          ,
%                            Im(f)           ,
%                            f(x)            ,
%                            f^{-1}(x)       ,
%                            f\restriction X ,
%                            \bigcup x       ,
%                            X\times Y       ,
%                            x+1             ,
%                            V_\omega        ,
%                            ||A\vDash_f\varphi||.
%                        $
%                        }
%                    \end{lemma}
                \end{theorem}
                \begin{theorem}{$V\vDash_f\varphi^A\BIM A\vDash_f\varphi$}
                    As hipóteses são $f\in A^{<\omega}$, $A\subseteq V$ transitivo e $V\vDash\ZF$.
                    \paragraph{}
                        A prova é na complexidade de $\varphi$ e vamos provar apenas para o conectivo $\land$ e $\neg$ e 
                        para o quantificador existêncial $\exists$. Para fórmulas atômicas, temos que, pelo fato de $A$ 
                        ser estrutura transitivia, então $\LDel{}{0}$ são absolutas.
                    \paragraph{}
                        Suponha que para todo $\psi$ de complexidade $n$ ou menor vale o teorema, vamos provar o passo 
                        indutivo em três casos: Ou $\varphi\equiv\psi\land\sigma$; ou $\varphi\equiv\neg\psi$; ou 
                        $\varphi\equiv\exists x_i: \psi$, todas com $\psi$ de complexidade $n$ e $\sigma$ mais simples.
                    \paragraph{}
                        No caso $\land$, temos:
                        $$ 
                             V\vDash_f (\psi\land\sigma)^A \BIM 
                             V\vDash_f \psi^A\land\sigma^A \BIM 
                            (V\vDash_f \psi^A)\land(V\vDash_f \sigma^A) \BIM 
                            (A\vDash_f \psi)\landV(\vDash_f \sigma) \BIM 
                             A\vDash_f \psi\land\sigma
                        $$
                    \paragraph{}
                        No caso $\neg$, temos:
                        $$ 
                            V\vDash_f (\neg\psi)^A \BIM 
                            V\vDash_f \neg(\psi)^A \BIM 
                            \neg(V\vDash_f \psi^A) \BIM 
                            \neg(A\vDash_f \psi  ) \BIM 
                            A\vDash_f \neg\psi
                        $$
                    \paragraph{}
                        No caso $\exists$, temos:
                        $$
                            V\vDash_f(\exists x_i:\psi)^A\BIM
                            V\vDash_f\exists x_i\in A:\psi^A \BIM
                            V\vDash_f\exists x_i:x_i\in A\land\psi^A \BIM
                        $$
                        $$
                            \BIM\exists \hat x\in V: (V\vDash_{f_{[x_i\slash\hat x]}}x_i\in A\land\psi^A) \BIM \exists \hat x\in V :[(V\vDash_{f_{[x_i\slash\hat x]}}x_i\in A)\land(V\vDash_{f_{[x_i\slash\hat x]}}\psi^A)]\BIM
                        $$
                        $$
                            \BIM\exists \hat x\in A: (V\vDash_{f_{[x_i\slash\hat x]}}\psi^A) \BIM \exists \hat x\in A: (A\vDash_{f_{[x_i\slash\hat x]}}\psi) \BIM A\vDash_f\exists x_i:\psi
                        $$
                        \eop
                \end{theorem}
                \paragraph{}
                    Isso nos dá a profundidade da relação das estruturas transitivas de $\ZF$ com um modelo base. Uma estrutura transitiva modela 
                    exatamente o que a sua estrutura ambiente crê que ela modela.
    \section{O Universo Construtível}
            Um modelo especial de $\ZF$ é o Universo Construtível, que satisfaz uma restrição adicional sobre sua estrutura fina. É um exemplo 
            de modelo onde vale o axioma da escolha e a hipótese generalizada do contínuo, e gostaríamos de passar por ele justamente para 
            contrapôr os modelos a valores Booleanos.
        \paragraph{}
            Chamamos o universo construtível de $L$, que é uma classe transitiva da hierarquia acumulada $V$. A definição de $L$ dentro da 
            língua depende de uma internalização da lógica e da teoria dos modelos transitivos. Mas, moralmente, queremos:

        $$ L_{\alpha+1} = \SET{x : \exists \varphi \text { fórmula da teoria: }\exists \vec{p} \in L_\alpha^{<\omega}: x = \SET{t\in V_\alpha : \varphi(t; \vec{p})}} $$
        $$ (\lambda = \bigcup \lambda) \RAR L_\lambda = \bigcup_{\alpha<\lambda}L_\alpha $$

        \paragraph{}
            Evidentemente, se ``$\varphi \text{ fórmula da teoria}$'' não estiver formalizado \textit{dentro} da teoria, não 
            temos esperaça alguma desta definição fazer sentido. Pois, o primeiro passo que devemos tomar é achar uma Gödelização
            apropriada das fórmulas como \textit{conjuntos} de fato. 
        \paragraph{}
            Suponha que possuamos códigos para $\FMLA{\forall}$, $\FMLA{\exists}$, $\FMLA{\!\!=\!\!}$, $\FMLA{\!\!\in\!\!}$, 
            $\FMLA{\neg}$, $\FMLA{\!\lor\!}$, $\FMLA{\!\land\!}$ e $\FMLA{\!\!\rar\!\!}$.
            Então, definimos a codificação das fórmulas da seguinte maneira, similar a de \cite{Drake}:

        \begin{align*}
            \FMLA{\texttt{Q} x_i:\varphi} &= \TUPLE{\FMLA{\texttt{Q}}}{i}{\FMLA{\varphi}}           \text{, onde \texttt{X} for um quantificador;}  \\
            \FMLA{x_i  = x_j}             &= \TUPLE{\FMLA{\!\!=\!\!}}{i}{j}                                                                         \\
            \FMLA{x_i\in x_j}             &= \TUPLE{\FMLA{\!\!\in\!\!}}{i}{j}                                                                       \\
            \FMLA{\neg\varphi}            &= \TUPLE{\FMLA{\neg}}{\FMLA{\varphi}}{\FMLA{\varphi}}                                                    \\
            \FMLA{\varphi\texttt{X}\psi}  &= \TUPLE{\FMLA{\texttt{X}}}{\FMLA{\varphi}}{\FMLA{\psi}} \text{, onde \texttt{X} for um conectivo;}      
        \end{align*}

        \paragraph{}
            Fixar códigos para os componentes simples não poderia ser mais fácil: temos apenas 8 deles, e $8 = \SET{\emptyset}{1}{2}{3}{4}{5}{6}{7}$.
            Não desejamos impor a nossa bijeção favorita, qualquer uma serve. Agora que temos uma especificação de como é uma ``fórmula'' internalizada, 
            podemos escrever uma fórmula \textit{de fato} que afirma que um dado \textit{conjunto} é uma \textit{representação} de um fórumla.
            
        \begin{definition}{Construção de uma Fórmula}
            \begin{align*}
                Repr(\varphi, \chi, n) \equiv &[n \in \omega] \land [Fun(\chi)] \land [Dom(\chi) = n+1] \land [\chi(n) = \varphi] \land \forall k\in n+1:\{\\
                    &[\exists a,b  \in k     : [\chi(k) = \TUPLE{ \FMLA{\!\!\rar\!\!} }{\chi(a)}{\chi(b)}] ] \lor \\
                    &[\exists a,b  \in k     : [\chi(k) = \TUPLE{ \FMLA{\neg}         }{\chi(a)}{\chi(a)}] ] \lor \\
                    &[\exists a,b  \in k     : [\chi(k) = \TUPLE{ \FMLA{  \!\lor\!  } }{\chi(a)}{\chi(b)}] ] \lor \\
                    &[\exists a,b  \in k     : [\chi(k) = \TUPLE{ \FMLA{  \!\land\!  }}{\chi(a)}{\chi(b)}] ] \lor \\
                    &[\exists a,i  \in k     : [\chi(k) = \TUPLE{ \FMLA{\forall}      }{   i   }{\chi(a)}] ] \lor \\
                    &[\exists a,i  \in k     : [\chi(k) = \TUPLE{ \FMLA{\exists}      }{   i   }{\chi(a)}] ] \lor \\
                    &[\exists i, j \in \omega: [\chi(k) = \TUPLE{ \FMLA{\!\!\in\!\!}  }{   i   }{   j   }] ] \lor \\
                    &[\exists i, j \in \omega: [\chi(k) = \TUPLE{ \FMLA{\!\! = \!\!}  }{   i   }{   j   }] ] \}
            \end{align*}
        \end{definition}

        \paragraph{}
            Apesar de longa, $Repr$ é bem simples: ela afirma que $\chi$ é testemunha da construção de $\varphi$ em $n$ passos. Fica claro que 
            um determinado conjunto $X$ é fórmula se e só se $\exists n\in\omega:\exists \chi \in V_\omega: Repr(X, \chi, n)$. 

        \paragraph{}
            Agora, deixe:

        \begin{definition}{Satisfatibilidade Internalizada}
            \begin{align*}
                Sat(A,\varphi,f)\equiv&\exists w:\exists \chi,n,r\in V_\omega:[Repr(\varphi,\chi,n)\land Fun(w)\land (Dom(w) = n+1)\land\\
                    &\land [r = rank(\varphi)] \land [f\in w(n)]\land\forall k \in n+1:[\\
                            [&\exists i, j\in\omega     : \chi(k) = \TUPLE{\FMLA{\!\!=\!\!}}{  i    }{    j  } \land w(k) = \SET{f \in A^r : f(i)  = f(j)}]\lor\\
                        \lor[&\exists i, j\in\omega     : \chi(k) = \TUPLE{\FMLA{\!\!\in\!\!}}{i    }{    j  } \land w(k) = \SET{f \in A^r : f(i)\in f(j)}]\lor\\
                        \lor[&\exists a, b\in   k       : \chi(k) = \TUPLE{\FMLA{\!\lor\! }}{\chi(a)}{\chi(b)} \land w(k) = w(a)\cup w(b)]\lor\\
                        \lor[&\exists a, b\in   k       : \chi(k) = \TUPLE{\FMLA{\!\land\!}}{\chi(a)}{\chi(b)} \land w(k) = w(a)\cap w(b)]\lor\\
                        \lor[&\exists a, b\in   k       : \chi(k) = \TUPLE{\FMLA{\!\rar\!}}{\chi(a)}{\chi(b) } \land w(k) = (A^r - w(a))\cup w(b)]\lor\\
                        \lor[&\exists a, b\in   k       : \chi(k) = \TUPLE{\FMLA{\!\neg\!}}{\chi(a)}{\chi(a) } \land w(k) = (A^r - w(a))]\lor\\
                        \lor[&\exists a\in k, i\in\omega: \chi(k) = \TUPLE{\FMLA{\exists }}{   i   }{\chi(a) } \land w(k) = \SET{v\in A^r : \exists x\in A: v_{[i\slash x]}\in w(a) }]\lor\\
                        \lor[&\exists a\in k, i\in\omega: \chi(k) = \TUPLE{\FMLA{\forall }}{   i   }{\chi(a) } \land w(k) = \SET{v\in A^r : \forall x\in A: v_{[i\slash x]}\in w(a) }]]]
            \end{align*}
            
            \paragraph{}
                Onde $v_{[i\slash x]} = v - \TUPLE{i}{v(i)} \cup \SET{\TUPLE{i}{x}}$, ou seja, substituição do valor de $v$ em $i$ por $x$.
            \paragraph{}
                $Sat(A,\varphi,f)$ quer dizer, essencialmente, que existe uma sequência de conjuntos de testemunhas para a veracidade 
                da fórmula e $f$ é uma das testemunhas.
        \end{definition}
        \begin{lemma}{Resultados sobre $Sat$}
            Se $f\in A^{<\omega}$, $A$ estrutura transitiva e $\varphi,\psi$ forem fórmulas. 
            $$Sat[A,\FMLA{\varphi}, f]\land Sat[A,\FMLA{\psi}, f] \bim Sat[A,\FMLA{\varphi\land\psi}, f]$$
            $$Sat[A,\FMLA{\varphi}, f]\lor  Sat[A,\FMLA{\psi}, f] \bim Sat[A,\FMLA{\varphi\lor \psi}, f]$$
            $$Sat[A,\FMLA{\varphi}, f]\rar  Sat[A,\FMLA{\psi}, f] \bim Sat[A,\FMLA{\varphi\rar \psi}, f]$$
            $$\neg Sat[A,\FMLA{\varphi}, f]                       \bim Sat[A,\FMLA{\neg\varphi     }, f]$$
            $$\exists\hat x\in A: Sat[A,\FMLA{\varphi}, f_{[x_i\slash\hat x]}]\bim Sat[A,\FMLA{\exists x_i:\varphi}, f]$$
            $$\forall\hat x\in A: Sat[A,\FMLA{\varphi}, f_{[x_i\slash\hat x]}]\bim Sat[A,\FMLA{\forall x_i:\varphi}, f]$$
            \begin{proof}
                    Para $\land$, suponha que $Sat$ valha para $\varphi$ e para $\psi$. Que existe a construção é trivial, para $w$, 
                    basta concatenar as $w$-s que existem e na última etapa interceptar a os últimos valores dos $w$-s. Por outro 
                    lado, se valer para $\varphi\land\psi$ então restringir as construções e $w$ é praticamente trivial.
                \paragraph{}
                    Para $\lor$, a ida é igual é a mesma, mas une-se ao invés de se interceptar. Para a volta, sabemos que $f$ está 
                    em alguma etapa final das duas $w$ que os $Sat$s nos dão, assim sabemos que $f$ estará na união, e portanto 
                    valerá $Sat(A,\FMLA{\varphi\lor \psi}, f)$.
                \paragraph{}
                    Para $\rar$, ao invés de se unir as etapas finais de testemunhas, unimos o complementar de uma com a outra. Para 
                    a volta, sabemos $f$ ou não é testemunha $\varphi$, ou é testemunha de $\psi$, então fica claro que vale a volta.
                \paragraph{}
                    Para $\neg$, se $f$ não valida $Sat(A,\FMLA{\varphi}, f)$, então é porque $f$ não está entre as testemunhas da 
                    última etapa de $w$. Isto é se e só se $f$ estiver no complementar das testemunas que é se e somente se   
                    $Sat(A,\FMLA{\neg\varphi}, f)$.
                \paragraph{}
                    Para o existencial, se existe um elemento de $A$ que faz $f$ estar na testemunhas, então pela definição de $Sat$
                    temos que valerá o $Sat$ de $\FMLA{\exists x_i:\varphi}$. E, adicionalmente, vale a volta. Similarmente vale o 
                    mesmo para o quantificador universal.
                    \eop
            \end{proof}
        \end{lemma}
        \begin{theorem}{$A\vDash_f\varphi \BIM V\vDash Sat(A,{}^\ulcorner{\varphi}^\urcorner,f)$}
            As hipóteses são: seja $A\subset V$ uma estrutura transitiva, e 
            $V\vDash \ZF$, $\varphi$ uma fórmula da língua de $\ZF$ e $f\in A^{<\omega}$. 
            Novamente a prova é por indução na complexidade da 
            fórmula. Novamente, o caso atômico é simples e, por isso, não o 
            faremos aqui. Provaremos o caso da conjunção, negação e quantificador 
            existêncial.
            \begin{proof}
                \paragraph{}
                    Seja $A$ uma estrutura transitiva e seja sempre $f$ uma $A$-valoração 
                    das variáveis pertinententes. A prova é por indução na complexidade 
                    de $\varphi$. Para fórmulas atômicas temos que a igualdade e pertinência 
                    em $A$ são só as restrições das relações à classe, então $V\vDash_f x = y$ exatamente 
                    quando $Sat(A,\FMLA{x = y},f)$ e similarmente para $(x\in y)$.
                \paragraph{}
                    Assim, assumamos que valha a bi-implicação para o caso de fórmulas de complexidade 
                    até $n$. Seja agora, $\psi$ e $\sigma$ fórmulas de complexidade até $n$.
                \paragraph{}
                    Vamos provar a bi-implicação da etapa sucessora só para $\land,\neg,\exists$ pois 
                    é suficiente, afinal $\vDash$ é muito bem comportado. Para $\land$ temos:
                $$A\vDash_f\psi\land\sigma \BIM (A\vDash_f\psi)\land(A\vDash_f\sigma)\BIM Sat(A,\FMLA{\psi},f)\land Sat(A,\FMLA{\sigma},f)\BIM Sat(A,\FMLA{\psi\land\sigma},f)$$
                \paragraph{}
                    Para o caso de  $\neg$,
                $$A\vDash_f\neg\psi\BIM\neg(A\vDash_f\psi)\BIM\neg Sat(A,\FMLA{\psi},f)\BIM Sat(A,\FMLA{\neg\psi},f)$$
                \paragraph{}
                    E no caso do quantificador $\exists$,
                $$A\vDash_f\exists x_i:\psi\BIM\exists\hat x\in A:(A\vDash_{f_{[x_i\slash\hat x]}}\psi)\BIM\exists\hat x\in A:Sat[A,\FMLA{\varphi}, f_{[x_i\slash\hat x]}]\BIM Sat[A,\FMLA{\exists x_i:\varphi}, f]$$
                \eop
            \end{proof}
        \end{theorem}
        \paragraph{}
            Com este teorema, temos uma correspondência muito forte entre: \textit{O ambiente achar que uma susbtrutura transitiva modela algo}, a
            \textit{substrutura transtitiva modelar algo} e a \textit{codificação da noção interna de modelos de fórmulas Gödelizadas}.

        \begin{definition}{L}
            Com isso, temos o suficiente para expressar a classe dos Construtíveis,
            $$ L_{\alpha+1} = \SET{x : \exists \varphi:\exists r\in\omega:\exists f\in L_\alpha^r: x = \SET{t\in L_\alpha : Sat(A,\varphi,f_{[0\slash t]})}} $$ 
            $$ (\lambda     = \bigcup\lambda) \rar L_\lambda = \bigcup_{\alpha < \lambda} L_\alpha $$ 
            $$ L = \bigcup_{\alpha\in Ord} L_\alpha $$
        \end{definition}
        \begin{definition}{Rank Construtível}
            Seja $\rho_c(x)=\min\SET{\alpha\in Ord: x\in L_\alpha}$, para todo $x$ que ocorre em $L$ este chamado Rank Construtível está definido.
            É claro que $\rho_c$ é homomorfismo, isto é, preserva pertinência. Ainda não podemos dizer o que ele faz com ordinais, por exemplo.
        \end{definition}
        \paragraph{}
            O primeiro resultado importante sobre os Construtíveis é, claro, que $L$ é modelo de $\ZF$, que iremos verificar a seguir. Na sequência, iremos verificar 
            que $L \vDash \ZF+AC+CH$. Que é grande coisa.

        \begin{lemma}{$L$ é transitivo}
            Basta ver que $L_\alpha$ é transitivo. Se $x\in L_{\alpha+1}$ então $\exists\varphi\in V_\omega\exists f\in L_\alpha^{<\omega}: x 
            = \SET{t\in L_\alpha : Sat(A,\varphi,f_{[0\slash t]})}$, oras então certamente $x \subseteq L_\alpha$. Similarmente, 
            se $\lambda$ for ordinal limite, $L_\lambda$ será união de transitivos, e portanto transitivo. Unindo $L_\alpha$ para todos os 
            os ordinais, temos uma classe transitiva.
        \end{lemma}
        \begin{lemma}{Conjunto contido em $L$ está contido em um membro de $L$}
            Seja $X\subset L$ um conjunto em $V$. Por substituição, o é conjunto $\SET{\alpha: \exists x\in X:\alpha = \rho_c(x)}$. 
            Como é conjunto de ordinais, ele possuí um supremo, que chamaremos de $\gamma$, então, por definição, todos os membros 
            de $x$ ocorrem em $L_\gamma$, assim, $X\subset L_\gamma\in L_{\gamma+1}$. Está provado.
        \end{lemma}

        \begin{theorem}{$L\vDash\ZF$}
                Provaremos que $L$ modela: O axioma do Vazio; O axioma do Par; O axioma 
                da União; Os axiomas da Substituição; O axioma do Infinito; O axioma das 
                Partes; O axioma da Fundação; O axioma da Extensionalidade. É largamente 
                conhecido que o esquema de Separação pode ser obtido através dos outros.
            \paragraph{Extensionalidade e Fundação}
                Pelo fato que $L$ é estrutura transitiva, temos, de graça, os axiomas de 
                Extensionalidade e Fundação.
            \paragraph{Axioma do Vazio}
                $$L\vDash\exists x:x=x\BIM V\vDash(\exists x:x=x)^L\BIM\exists x\in L: x = x\bim L\not = \emptyset$$
                É claramente o caso de $L$ não ser vazio, pois $L_1$ não é vazio. Aqui entra com força os meta-teoremas 
                que provamos sobre fórmulas relativizadas e $\vDash$.
            \paragraph{Axioma do Par}\nl
                Queremos
                $$L\vDash\forall x:\forall y:\exists z:\forall t: t\in z \bim (t = x\lor t = y)$$
                Que será o caso exatamente quando 
                $$V\vDash\forall x\in L:\forall y\in L:\exists z\in L:\forall t\in L: t\in z \bim (t = x\lor t = y) $$
                Mas é claro que $V\vDash\ZF$, então vamos apenas provar a fórmula acima em $\ZF$.
            \paragraph{}
                Sejam $x,y\in L_{\alpha}$ seja $z = \SET{t\in L_{\alpha}: Sat(L_\alpha, \FMLA{x_0 = x_1 \lor x_0 = x_2}, f)}$
                onde $f=\SET{\TUPLE{\FMLA{x_0}}{t}}{\TUPLE{\FMLA{x_1}}{x}}{\TUPLE{\FMLA{x_2}}{y}}$, que, por razões óbvias, vamos 
                abreviar como $[t,x,y]$ e continuaremos tratando valorações desta forma. Fica claro que com esta valoração 
                $z$ está em $L_{\alpha+1}$, e portanto em $L$. Não é difícil ver que $z$ é de fato o par em $L$.
            \paragraph{Axioma da Soma}\nl
                Queremos 
                $$L\vDash\forall x\exists y:\forall z: z\in y \bim \exists t: t\in y\land z\in t$$
                Deixe, $x\in L$, logo em $L_\alpha+1$ para algum $\alpha$. Deixe agora $\varphi\equiv\exists x_2\in x_1:x_0\in x_2$, e 
                $$ y = \SET{t\in L_\alpha : Sat(L_\alpha, \FMLA{\varphi}, f_{[x_0\slash z, x_1\slash x]})}$$

                Sabemos que $V\vDash_f y\in L$ e que $V\vDash_f\forall t: t\in y\bim\exists y\in x: t\in y$, então segue que 
                $V\vDash_f(\forall t: t\in y\bim\exists y\in x: t\in y)^L$, pois o universal restringe-se e o existencial 
                já está em $L$ pois $x$ está, e temos transitividade. Logo, $L\vDash_f\forall t: t\in y\bim\exists y\in x: t\in y$
                que era o que queríamos: uma testemunha da validade do axioma para cada $x$ dado.
            \paragraph{Axioma das Partes}\nl
                Queremos
                $$L\vDash\forall x:\exists y:\forall z:(\forall t\in z\rar t\in x)\rar z\in y$$
                Seja $x\in L$, e deixe $y = \SET{z\in L: (z\subseteq x)^L}$. Vemos que $y$ é 
                claramente um conjunto, resta verificar que ele está em $L$ e teremos o que 
                buscamos. Seja  
                $$\beta = \bigcup_{z\in y}\rho_c(z)+1$$
                \paragraph{}
                    Então sabemos que $y = \SET{z\in L_\beta : (z\subseteq x)^L}$, sabemos também que 
                    $\rho_c(x)<\beta$ afinal $x\in y$. Como $L_{\rho_c (x)}\subset L_\beta\subset L$, 
                    então uma é subestrutura transitiva da outra. Assim, $y = \SET{z\in L_\beta : 
                    (z\subseteq x)^{L_\beta}}$, que sabemos ser meta-equivalente a
                $$y = \SET{z\in L_\beta : Sat(L_\beta, \FMLA{x_0\subseteq x_1}, [z, x])}$$
                \paragraph{}
                    Então $y\in L$, e então está provado.
            \paragraph{Axioma da Substituição}\nl
                Queremos
                $$L\vDash[\forall x, y, z:\psi(y, x)\land\psi(z,x)\rar z=y]\rar[\forall x:\exists X:\forall y: y\in X\bim\exists t\in x:\psi(y,t)]$$
                Suponha uma fórmula $\psi$ nas condições do axioma. Deixe $x\in L$ e $X=\SET{y\in L:(\psi(y,x))^L}$

        \end{theorem}
        


%\paragraph{}
%    Sejam $a, b\in L$, pela definição de $L$, $\exists \alpha,\beta\in Ord:(a\in L_\alpha)\land(b\in L_\beta)$. Com isso, seja $\gamma$ o maior entre 
%    eles. Como $a,b\in L_\gamma$ então $\SET{t\in L_\gamma: Sat(L_\gamma, \FMLA{t = y \lor t = x},f)}$ com $f(x) = a \land f(y) = b$ claramente nos dá que 
%    o par de dois conjuntos está em $L_{\gamma+1}$, como provando o Axioma do Par.
%\paragraph{}
%    Para o Axioma da Soma, seja $X\in L$, então $X\in L_{\rho_c(x)}$, por substituição em $\ZF$ temos que $R = \SET{\alpha : \exists x\in X:\alpha = \rho_c(x)}$ 
%    é conjunto. Seja $\gamma = \sup(R)$, Então em $L_\gamma$ ocorrem todos os elementos de $x$, e pela transitividade de $L_\gamma$, todos os elementos deles.
%    Considere então a fórmula $\varphi(t; X)\equiv \exists x\in X: t\in x$, ela dá $\bigcup X$ em $L_{\gamma+1}$ quando tomando parâmetro $X$.
%\paragraph{}
%    Para o axioma das partes, seja $x\in L$, então $x\in L_{\alpha+1}$ para algum $\alpha$ conveniente. Então existe uma $\psi$ fórmula interna, e $f$ 
%    parâmetros tal que $x = \SET{t \in L_\alpha : Sat(L_\alpha, \psi, f)}$, então $x\subseteq L_\alpha$. Assim, considere a fórmula $\varphi(
%    t;x,f,\psi,\alpha)\equiv\exists g\in L_\alpha^{<\omega}:\exists\phi:t=\SET{y\in L_\alpha:Sat(L_\alpha,\phi,g)\land Sat(L_\alpha,\psi,f)}$.
%\paragraph{}
%    Haverá parâmetros para esta fórmula em algum $L_\beta$ que nos dê algo similar a $\wp(x)$? Em $(\alpha+2\omega)$ com certeza haverá, pois, 
%    lá já teremos todas as $f$ parâmetros em $L_\alpha$, todas as fórmulas, $x$ e $\alpha$, certamente. E como convencer-nos de que será realmente 
%    as partes de $x$? Se $(y\subset x)^ L$, então há uma fórmula que concretiza $y$ no nível que $x$ ocorre que é mais específica que 
%    a fórmula que seleciona $x$. Não perdemos, assim, generalidade em escrever $[\wp(x)]^ L$ como sendo aquela fórmula.
%\paragraph{}
%    Quanto ao Axioma do Infinito, sabemos que o vazio está em $L$, vamos assumir que até um ordinal $\alpha$, todos os ordinais finitos estão. Eles
%    aparecem em uma etapa $\beta$ da construção do $L$. Sabemos que o par $\SET{\alpha}{\alpha}$ existe na etapa $\beta+1$, que o par $\SET{\alpha}{\SET{\alpha}{\alpha}}$
%    existe na etapa $\beta+2$ e que em alguma etapa existe $\bigcup\SET{\alpha}{\SET{\alpha}{\alpha}}$, que é justamente o sucessor de $\alpha$. Logo,
%    existe um conjunto indutivo em $L$, pois ser indutivo é $\LDel{}{1}$, isto é, ser indutivo ``aqui fora'' é o mesmo que ser indutivo ``lá dentro''. 
%\paragraph{}
%    Sejam $X\in L_\alpha+1$ e $\psi$ uma fórmula da língua de $\ZF$ nas hipóteses do Esquema de Substituição. 
%    Isto é, $\forall y,z:\psi(y;x)\land\psi(z;x)\rar y = z$. Temos que existe um conjunto 
%    $Y   = \SET{y: \exists x\in X: \psi(y,x)}$. Se relativizarmos a fórmula, teremos ainda um conjunto:
%    $Y' = \SET{y: \exists x\in X: \psi(y,x)^ L}$. Este conjunto não tem obrigação nenhuma de 
%    estar em $L$ (tome, por exemplo, $\psi$ que só aceita $y$ fora de $L$), mas $Y_c=Y'\cap L$ tem.