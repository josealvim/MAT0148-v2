\chapter{Lógica de Primeira Ordem}
    \epigraph{\justify
        It seems to me now that mathematics 
        is capable of an artistic excellence 
        as great as that of any music 
        \elide because it gives in absolute 
        perfection that combination, 
        characteristic of great art, of godlike 
        freedom, with the sense of inevitable 
        destiny; because, in fact, it constructs 
        an ideal world where everything is 
        perfect and yet true.
        }{\textit{Bertrand Russell.}}    
    \cls
    \paragraph{}
        Começamos com uma língua $\mathcal{L}$ de primeira ordem, isto é, 
        a coleção de palavras (bem-)formadas por um determinado conjunto de 
        símbolos para variáveis; uma série de símbolos para relações,
        $\TUPLE{R_1}{R_2}{\ldots}$; uma de funções, $\TUPLE{F_1}{F_2}{\ldots}$
        e uma de constantes $\TUPLE{C_1}{C_2}{\ldots}$; conectivos e operadores 
        lógicos e quantificadores.
    \begin{definition}{Termos, Fórmulas e variáveis livres ${\vert}$ escopadas}
        \begin{enumerate}[label=(\alph*)]
            \item Se $x_i$ for uma variável, então a palavra ``$x_i$'' é um 
                \strong{termo} e a (única) ocorrência de $x_i$ é dita 
                \strong{livre} no termo.

            \item Se $C_i$ for uma constante, então a palavra ``$C_i$'' é 
                um \strong{termo}. Como ele não contém variável, não faz 
                sentido tratar de ocorrências de variáveis livres ou escopadas.

            \item Se $\SET{t_1}{\ldots}{t_k}$ forem termos, e $R_j$ é um 
                relacional de aridade $k$, então a palavra 
                ``$(R_j\POINT{t_1}{\ldots}{t_k})$'' é uma \strong{fórmula}. E 
                cada ocorrência de cada variável presente nesta é dita livre 
                nela exatamente quando for livre no termo em que ocorre e 
                é dita escopada exatamente quando é escopada no termo que 
                ocorre.
        
            \item Se $\SET{t_1}{\ldots}{t_k}$ forem termos, e $F_j$ é um 
                relacional de aridade $k$, então a palavra 
                ``$(F_j\POINT{t_1}{\ldots}{t_k})$'' é um \strong{termo}. E 
                cada ocorrência de cada variável presente nesta é dita livre 
                nela exatamente quando for livre no termo em que ocorre e 
                é dita escopada exatamente quando é escopada no termo que 
                ocorre.

            \item Se $\varphi,\psi$ forem fórmulas e $\diamond$ for um conectivo 
                lógico, então as palavras: ``($\varphi\diamond\psi$)'' e ``($\neg\varphi$)''
                são \strong{fórmulas}, e occorências de variáveis são livres ou escopadas 
                de acordo com sua liberdade ou limitação nas fórmulas componentes.
            
            \item Se $\varphi$ for uma fórmula, e $x_i$ é uma variável, então as 
                palavras: ``$\exists x_i:\varphi$'' e ``$\forall x_i:\varphi$''
                são \strong{fórmulas} e cada ocorrência de $x_i$ é dita \strong{escopada}
                --- pois está no escopo de um quantificador ---,
                e cada ocorrência de qualquer outra variável é livre ou escopada 
                de acordo com sua situação em $\varphi$.

%            \item Se $\varphi$ for uma fórmula e $x_i$ for uma variável, então 
%                a palavra ``$\SET{x_i \vert \varphi }$'' é um \strong{termo} 
%                com todas ocorrências de $x_i$ escopadas e outras ocorrências como em 
%                $\varphi$.
        \end{enumerate}
        \paragraph{}
            A língua é formada destes componentes, fórmulas e termos.
    \end{definition}
    
    \paragraph{}
        Por legibilidade, por omitiremos parenteses quando possível, e 
        intruduziremos abreviaturas para fórmulas, termos \emph{etc.}
    \paragraph{}
        A ideia de uma língua é que ela \emph{fale sobre} um determinado
        domínio. Quando escrevo, minhas palavras tratam de objetos de alguma 
        sorte. Quando nomeio um objeto e predico sobre ele, por exemplo, 
        existe uma interpretação da verdade do predicado no nosso universo.
    \paragraph{}
        Por exemplo, ``\emph{Eu tenho cinco dedos cada uma de minhas mãos}'',
        é uma sentença da língua portuguesa, cujo valor de verdade é dependente
        em quem é ``\emph{Eu}'' e em suas propriedades. Poderia ser o caso que
        o autor {\emph{não}} tivesse cinco dedos em cada uma de minhas mãos, então,
        a interpretação costumeira de quem \emph{é} ``Eu'' falsearia a sentença.
    \paragraph{}
        Poderia ser o caso que o autor não tivesse \emph{mão alguma}, mas neste 
        caso a sentença seria verdade, pelo mesmo motivo que eu tenho uma 
        lamborghini em cada uma das minhas mansões.
    \paragraph{}
        Tentando capturar a noção de interpretação das \emph{palavras} para 
        objetos concretos (que não pertencem à língua, mas a um mundo) --- 
        por exemplo ``\emph{Eu}'' --- e como, associando palavras a objetos e 
        símbolos a relações, constantes, funções \emph{etc.} temos como 
        determinar a veracidade de uma proposição no mundo alvo, dada uma 
        interpretação.
    
    \begin{definition}{Estrutura, Interpretação e Validação}
        \paragraph{Assinatura:}
            Uma \strong{assinatura} é uma tripla de coleções: a das 
            relações, cada qual com sua aridade; 
            a das constantes; e, finalmente, a das funções, 
            cada qual com sua aridade.
        $$\Sigma=\TUPLE{
            \TUPLE{
                \mathbf{R_1}
            }{
                \ldots}
        }{
            \TUPLE{
                \mathbf{C_1}
            }{
                \ldots
        }}{
            \TUPLE{
                \mathbf{F_1}
            }{
                \ldots
        }}$$
        \paragraph{}
            Se cada relação ou função em uma assinatura $\Sigma$ 
            tiver domínio em uma coleção $A$, e cada constante 
            estiver em $A$, então podemos dizer que $\Sigma$ está 
            definida sobre $A$. 

        \paragraph{Estrutura:}
            Chamamos a dupla $\TUPLE{A}{\Sigma}$ de $A$ coleção e uma assinatura 
            $\Sigma$ definida sobre $A$, de $\mathfrak{A}$. E este tipo de objeto batizamos de \strong{estrutura}. 
            E dizemos $A$ ser o domínio de $\mathfrak{A}$, que escrevemos $|\mathfrak{A}|$, 
            e $\Sigma$ ser assinatura de $\mathfrak{A}$, possivelmente $sgn(\mathfrak{A})$.

        \paragraph{Compatibilidade Estrutura-Língua:}
            Dada uma língua de primeira ordem $\mathcal{L}$ e uma estrutura $\mathfrak{A}$ 
            dizemos elas serem compatíveis se e só se valem todas:
            \begin{enumerate}[label=(\alph*)]
                \item Se $sgn(\mathfrak{A})$ tiver tantas relações quanto há relacionais na língua.
                \item Se $\mathbf{R_i}$ é a $i$-ésima relação da estrutura, então\\
                $R_i$, o $i$-ésimo símbolo relacional da língua, concorda com sua 
                aridade.
                \item Se $sgn(\mathfrak{A})$ tiver tantas  funções quanto há funcionais  na língua.
                \item Se $\mathbf{F_i}$ é a $i$-ésima função da estrutura, então\\
                    $F_i$, o $i$-ésimo símbolo funcional da língua, concorda com sua 
                    aridade.
            \end{enumerate}
        
        \paragraph{Interpretação:}
            Dizemos que uma $f$ é uma interpretação de $\mathcal{L}$ para $\mathfrak{A}$
            se e, só se:
            \begin{enumerate}[label=(\alph*)]
                \item $\mathcal{L}$ e $\mathfrak{A}$ são compatíveis.
                \item $f$ associa cada variável $\lambda$ da língua para algum 
                    elemento $f(\lambda)$ de $|\mathfrak{A}|$.
                \item $f$ leva cada símbolo relacional   $R$ em uma relação $f[R]$ da $sgn(\mathfrak{A})$.
                \item $f$ leva cada símbolo funcional    $F$ em uma função  $f[F]$ da $sgn(\mathfrak{A})$.
                \item Se ${x_1},{\ldots},{x_k}$ forem termos e $F$ um funcional, então\\
                    $f(F\POINT{x_1}{\ldots}{x_k}) = f[F]\POINT{f(x_1)}{\ldots}{f(x_k)}$.
                \item $f$ leva cada símbolo de constante $c$ em um membro   $f(c)$ de $|\mathfrak{A}|$.
            \end{enumerate}
        \paragraph{}
            Isto é, ter uma interpretação --- ou melhor --- \emph{interpretar} um 
            símbolo é dar significado a ele. Pois, leva-lo a um objeto, relação 
            semântica entre objetos, função, \emph{etc}.
        \paragraph{}
            Um exemplo de significação é imaginar que quando escrevo ``\emph{o 
            autor}'' em alguma sentença, pode-se \emph{interpretar} como referente do 
            símbolo o \emph{autor deste texto}. Ficam, pois, definidas as propriedades 
            do signo e, assim, este pedaço da sentença ganha significado.
        \paragraph{}
            Além disso, dada uma interpretação $f$ de $\mathcal{L}$ para $\mathfrak{A}$,
            uma variável ou constante $x$ e um membro $a$ de $|\mathfrak{A}|$, 
            definimos $f_{[x\slash a]}$ como sendo exatamente igual a $f$ nas variáveis 
            e constantes exceto em $x$, onde ela valerá $a$. Os outros valores de $f$ 
            --- isto é, em termos diversos --- induz-se das funções, constantes e variáveis.
        \paragraph{}
            Isto se comporta como mudar um pedaço da intepretação, por exemplo, a 
            interpretação de ``\emph{o nome do autor tem a letra W}'' muda radicalmente 
            de acordo com quem é o autor, mesmo mantendo a interpretação da função ``nome 
            de X''. Não poderia ``\emph{autor}'' denotar Ludwig Wittgenstein, que foi 
            autor?

        \paragraph{Validação:}
            Definimos uma fórmula $\varphi$ de $\mathcal{L}$ valer em $\mathfrak{A}$ sob 
            a interpretação $f$ (estrutura compatível com a língua) --- alternativamente, 
            que $\mathfrak{A}$ satisfaz $\varphi$ ---, ou simbolicamente, $\mathfrak{A}
            \vDash_{f}\varphi$ por recorrência sobre sua complexidade.\\
        (a) Se $x_0,\ldots,x_{\rho_i}$ forem termos quaisquer, $R_i$ relacional de aridade $\rho_i$:
            $$\varphi\equiv R_i\POINT{x_0}{\ldots}{x_{\rho_i}}
                \RAR\left[\mathfrak A\vDash_f\varphi \BIM f[R_i]\POINT{f(x_0)}{\ldots}{f(x_{\rho_i})}\right]$$
        (b) Se $\sigma$ e $\psi$ forem fórmulas, $x$ uma variável:
            \begin{align*}
                \varphi\equiv \sigma\land\psi &\RAR\\
                    [\mathfrak{A}\vDash_f\varphi&\BIM\mathfrak A\vDash_f\sigma\text{ e  }\mathfrak A\vDash_f\psi]
            \end{align*}
            \begin{align*}
                \varphi\equiv \sigma\lor \psi &\RAR\\
                    [\mathfrak{A}\vDash_f\varphi&\BIM\mathfrak A\vDash_f\sigma\text{ ou }\mathfrak A\vDash_f\psi]
            \end{align*}
            \begin{align*}
                \varphi\equiv \neg\sigma      &\RAR\\
                    [\mathfrak{A}\vDash_f\varphi&\BIM\text{ não }\mathfrak A\vDash_f\sigma]
            \end{align*}
            \begin{align*}
                \varphi\equiv \sigma\rar\psi  &\RAR\\
                    [\mathfrak{A}\vDash_f\varphi&\BIM\mathfrak A\vDash_f\neg\sigma\text{ ou }\mathfrak A\vDash_f\psi]
            \end{align*}
            \begin{align*}
                \varphi\equiv \forall x:\sigma&\RAR\\
                    [\mathfrak{A}\vDash_f\varphi&\BIM\text{ para todo $a$ de    $|\mathfrak A|$: }
                    \mathfrak A\vDash_{f_{[x\slash{a}]}}\varphi]
            \end{align*}
            \begin{align*}
                \varphi\equiv \exists x:\sigma&\RAR\\
                    [\mathfrak{A}\vDash_f\varphi&\BIM\text{ existe algum $a$ de $|\mathfrak A|$: }
                    \mathfrak A\vDash_{f_{[x\slash{a}]}}\varphi]
            \end{align*}
        \paragraph{}
            A motivação desta definição é a ideia que os fatos elementares,
            no caso as relações e funções entre os objetos, determinam todos 
            os fatos ``complexos''. Os quantificadores por sua vez, desempenham o papel de providenciar
            genericidade. Isto é, sem estes, a validade de uma fórmula é 
            extremamente dependente da interpretação. Mas uma \strong{sentença} 
            --- isto é, uma fórmula com todas as variáveis em algum escopo de 
            quantificador ---, de certa forma, não depende de uma interpretação. 
    \end{definition}

    \begin{definition}{Sentença}
        Uma \strong{sentença} de uma língua é uma fórmula em que todas as ocorrências 
        de suas variáveis são escopadas.
    \end{definition}

    \begin{lemma}{Genericidade restrita de fórmulas:}
        Se todas as ocorrências de uma variável $x$ em uma dada fórmula $\varphi$ 
        estiverem escopadas, então para todo $a$ no domínio de uma certa estrutura 
        $\mathfrak{A}$ compatível com a língua de $\varphi$, $\mathfrak{A}\vDash_{f}
        \varphi\BIM\mathfrak{A}\vDash_{f_{[x\slash{a}]}}\varphi$.
    \end{lemma}
    \begin{proof}
        Onde $x$ aparecer, ela estará escopada, então \emph{ja íamos} substituir
        o valor de $f$ lá. Como a validade das fórmulas é dada por recorrência,
        e como tudo permanece o mesmo salvo em $x$, fica clara a ida, para a volta 
        reaplica-se a inda.\eop
    \end{proof}

    \begin{lemma}{Independência das interpretações irrelevantes:}
        Se $x$ não ocorrer em $varphi$ uma $\mathcal{L}$-fórmula
        com língua compatível com $\mathfrak{A}$ estrutura, então
        para todo $a$ em $|\mathfrak{A}|$, $\mathfrak{A}\vDash_{f}
        \varphi\BIM\mathfrak{A}\vDash_{f_{[x\slash{a}]}}\varphi$.
    \end{lemma}
    \begin{proof}
        Na definição de validação $x$ não desempenha papel algum.\eop
    \end{proof}

    \begin{theorem}{Genericidade de sentenças para validação:}
        Dadas uma língua e uma estrutura compatíveis. Se $\varphi$ for uma 
        sentença da língua, então a estrutura valida $\varphi$ sob uma 
        interpretação $f$ se e somente se valida sob qualquer interpretação 
        $g$ que leva as relações, funções e constantes nas mesmas que leva 
        $f$.
    \end{theorem}
    \begin{proof}
        A volta é trivial, se valida para qualquer que coincide com $f$ em 
        certa parte, valida para $f$, que de certo coincide com $f$ nesta 
        parte.
        \paragraph{}
            Por outro lado, todas as ocorrências de variáveis são escopadas,
            então, qualquer substituição de variáveis (presentes em $\varphi$) 
            na intepretação não altera a validade da sentença. Adicionalmente,
            qualquer substituição de variáveis irrelevantes não altera a 
            validade. Pois, quaisquer substituições não alteram a validade.
            Logo, toda interpretação leva a mesma validação.\eop
    \end{proof}

    \begin{definition}{Satisfatibilidade}
        Finalmente, definiremos uma noção de satisfatibilidade ou validade
        que independe de \emph{interpretação}.
        \paragraph{}
            Dadas $\mathfrak{A}$ uma estrutura e $\mathcal{L}$ uma língua 
            compatíveis, e uma fórmula $\varphi$ da língua.
        \paragraph{}
            Deixe $\psi\equiv\forall x_1,\ldots,\forall x_k:\varphi$ onde 
            $x_1,\ldots,x_k$ são todas as variáveis em $\varphi$ não escopadas,
            $$\mathfrak{A}\vDash\psi\BIM[\text{Para toda interpretação $f$}:\mathfrak{A}\vDash_{f}\psi]$$
        \paragraph{}
            Possívelmente, nenhuma variável será livre em $\varphi$
            neste caso, $\psi$ coincidirá com $\varphi$.
        \paragraph{}
            No caso que $\mathfrak{A}\vDash\varphi$ vale, dizemos que 
            $\mathfrak{A}$ satisfaz $\varphi$ ou, alternativamente,
            $\varphi$ é válida em $\mathfrak{A}$. Note que uma fórmula 
            que não for sentença é válida só se por falta de contra 
            exemplo.
    \end{definition}

    O estudo das estruturas e as fórmulas que lá são satisfeitas é 
    extremamente importante, afinal. De exemplos, um trivial é uma 
    estrutura com (apenas) a relação identidade e uma língua com 
    (apenas) o símbolo ``$=$'', sendo assim, temos que 
    $$ \TUPLE{A}{Id}\footnote{
        No caso, interprete que a dupla é uma estrutura com apenas relações, 
        nesta instância $Id$ e domínio $A$.
    }\vDash x = x$$


    \paragraph{Notação:}
        A partir deste ponto, não vamos explicitamente dizer que 
        uma língua é compatível com uma estrutura quando tratamos 
        delas. Nem diremos que uma $\varphi$ é fórmula da língua,
        \emph{etc.}

    \begin{definition}{Consequência}
        Da mesma forma que definimos satisfatibilidade com a intenção
        de abstrair a noção de interpretação, queremos agora abstrair
        as relações da própria estrutura.

        \paragraph{Estruturas compatíveis}
            Duas estruturas são ditas compatíveis exatamente quando 
            houver uma língua que é compatível com as duas, isto é 
            as aridades de cada relação, função, número de constantes
            \emph{etc.} coincidem.

        \paragraph{}
            Duas estruturas compatíveis podem satisfazer fórumlas 
            radicalmente diferentes, por exemplo, a estrutura que 
            tomamos acima e uma outra estrutura com apenas a 
            relação ``$x (Nq) y\BIM x\not=y$''. Se $\mathcal{L}$ for a língua 
            com apenas um símbolo relacional $R$, fica claro que 
            $\TUPLE{A}{Id}\vDash x = x$ e $\TUPLE{B}{Nq}\not\vDash x = x$.
            Queremos, pois, uma noção que não dependa tanto da estrutura fina
            do universo interpretativo.

        \paragraph{Consequência:}
            Dizemos que $\varphi$ segue, ou é consequência, de $\psi$, ou simbolicamente,
            $$ \varphi\vdash\psi \BIM [\text{Para toda }\mathfrak{A}:\mathfrak{A}\vDash\psi\RAR\mathfrak{A}\vDash\varphi]$$
            \paragraph{}
                Note a semelhança com a definição de satisfatibilidade
                sem interpretação. Novamente, a quantificação universal
                faz o trabalho de generalizar o conceito em certa direção.
            \paragraph{}
                Extendemos a definição para conjuntos de fórmulas dos dois 
                lados,
            $$ \Gamma\vdash\Sigma $$
                Da maneira natural de fazer, isto é, para cada $\varphi$ em $\Sigma$,
                $ \Gamma\vdash\varphi $, e por isso, queremos dizer:
            $$ [\text{Para toda }\mathfrak{A}:(\text{para toda $\gamma$ em $\Gamma$}:\mathfrak{A}\vDash\psi)\RAR\mathfrak{A}\vDash\varphi]$$
            \paragraph{}
                Finalmente, se $\emptyset\vdash\varphi$, escrevemos $\vdash\varphi$.
    \end{definition}
    \paragraph{}
        A interpretação da definição é que algumas fórmulas são genéricas sobre 
        uma certa sorte de estrutura, por exemplo, considere a fórmula $\varphi\equiv 
        xRx \rar xRx$. Não importa \emph{quem} seja $x$, nem quais 
        sejam suas relações com quaisquer outros membros de seu universo. 
        $\varphi$ é \emph{sempre} verdade.
    \paragraph{}
        Considere um outro exemplo menos formal do mesmo princípio: 
        Seja $\psi=$``\emph{Todos os cavalos são animais}'' e 
             $\varphi=$``\emph{todas as cabeças de cavalo são cabeças de animais}''
        então $\psi\vdash\varphi$ é válida. Isso não depende de se existem cavalos,
        o quê são cavalos, o quê são cabeças, se cavalos tem quatro patas, 
        se cavalos tem cabeças, se cavalos são uma cor \emph{etc.} E isso 
        não é jogo de palavras: 
        $$\psi\equiv\forall x: E(x,x)\rar A(x,x)$$ $$\varphi\equiv\forall c: (\exists a: E(a,a)\land C(c,a))\rar \exists b: A(a,a)\land C(c,b)$$
        \paragraph{}
            Que podemos ler ``$\varphi$: para todo objeto, se este for um Equino, 
            ele é um Animal'' e ``$\psi$: para todo objeto, se existe um outro 
            que é Equino e tem como Cabeça este, então existe um terceiro que é 
            Animal e tem como Cabeça o primeiro''.
        \begin{proposition}{Resultados sobre $\vdash$}
            \begin{enumerate}[label=\alph*)]
                \item $    {\varphi}\vdash\varphi$,
                \item $\Gamma\vdash\varphi$ e $\Gamma\subseteq\Lambda\RAR \Lambda\vdash\varphi$,
                \item $    {\varphi}\vdash\varphi\lor\psi$,
                \item $\SET{\varphi}{\psi}\vdash\varphi\land\psi$,
                \item $    {\varphi\land\psi}\vdash\SET{\varphi}{\psi}$,
                \item $    \vdash  \varphi\lor\neg\varphi$,
                \item $\varphi\vdash\neg\neg\varphi$,
                \item $\neg\neg\varphi\vdash\varphi$,
                \item $    {\varphi}\vdash\psi\BIM\vdash\varphi\rar\psi$,
                \item $    {\forall x:\varphi}\vdash\exists x:\varphi$,
                \item $    {\forall x:\varphi}\vdash\neg\exists x:\neg\varphi$,
                \item $    {\neg\exists x:\neg\varphi}\vdash\forall x:\varphi$,
                \item $\vdash\varphi\RAR{}\vdash\SET{\forall x:\varphi}{\exists x:\varphi}$,
                \item $\SET{\varphi}{\varphi\rar\psi}\vdash\psi$,
                \item $\SET{\varphi\lor\psi}{\neg\varphi}\vdash\psi$.
            \end{enumerate}
        \end{proposition}
        \begin{proof}
            \begin{enumerate}[label=\alph*)]
                \item $ $
                    \begin{prooftree}
                        \AxiomC{$\mathfrak{A}\vDash{\varphi}$}
                        \UnaryInfC{$\mathfrak{A}\vDash\varphi$}
                        \UnaryInfC{$\mathfrak{A}\vDash{\varphi}\RAR\mathfrak{A}\vDash\varphi$}
                        \UnaryInfC{${\varphi}\vdash\varphi$}
                    \end{prooftree}
                \item $ $
                    \begin{prooftree}
                        \AxiomC{$\Gamma\vdash\varphi$}
                        \UnaryInfC{$\mathfrak{A}\vDash\Gamma\RAR\mathfrak{A}\vDash\varphi$}

                        \AxiomC{$\Gamma\subseteq\Lambda$}
                        
                        \AxiomC{$\mathfrak{A}\vDash\Lambda$}
                        \UnaryInfC{$\lambda\in\Lambda\RAR\mathfrak{A}\vDash\lambda$}
                        
                        \BinaryInfC{$\gamma\in\Gamma \RAR\mathfrak{A}\vDash\gamma$}
                        \UnaryInfC{$\mathfrak{A}\vDash\Gamma$}
                    
                        \BinaryInfC{$\mathfrak{A}\vDash\varphi$}
                    \end{prooftree}
                    $$\therefore\mathfrak{A}\vDash\Lambda\RAR\mathfrak{A}\vDash\varphi$$
                    $$\text{Assim, } \Lambda\vdash\varphi$$

                \item $ $
                    \begin{prooftree}
                        \AxiomC{   $\mathfrak{A}\vDash{\varphi}$}
                        \UnaryInfC{$\mathfrak{A}\vDash{\varphi}$ ou $\mathfrak{A}\vDash{\psi}$}
                        \UnaryInfC{$\mathfrak{A}\vDash{\varphi\lor\psi}$}
                    \end{prooftree}                                            
                    $$\therefore{\varphi}\vdash\varphi\lor\psi $$

                \item $ $
                    \begin{prooftree}
                        \AxiomC{$\mathfrak{A}\vDash\SET{\varphi}{\psi}$}
                        \UnaryInfC{$\mathfrak{A}\vDash\varphi$}
                        \AxiomC{$\mathfrak{A}\vDash\SET{\varphi}{\psi}$}
                        \UnaryInfC{$\mathfrak{A}\vDash\psi$}
                        \BinaryInfC{$\mathfrak{A}\vDash\varphi$ e $\mathfrak{A}\vDash\psi$}
                        \UnaryInfC{$\mathfrak{A}\vDash\varphi\land\psi$}
                    \end{prooftree}
                    $$\therefore\SET{\varphi}{\psi}\vdash\varphi\land\psi $$
                
                \item $ $
                    \begin{prooftree}
                        \AxiomC{$\mathfrak{A}\vDash{\varphi\land\psi}$}
                        \UnaryInfC{$\mathfrak{A}\vDash{\varphi}$ e $\mathfrak{A}\vDash{\psi}$}
                        \UnaryInfC{$\mathfrak{A}\vDash\SET{\varphi}{\psi}$}
                    \end{prooftree}
                    $$\therefore{\varphi\land\psi}\vdash\SET{\varphi}{\psi} $$
                
                \item $ $
                    \begin{prooftree}
                        \AxiomC{$\mathfrak{A}\vDash\emptyset$}
                        \AxiomC{$\mathfrak{A}\not\vDash\varphi$}
                        \BinaryInfC{Não $\mathfrak{A}\vDash\varphi$}
                        \UnaryInfC{$\mathfrak{A}\vDash\neg\varphi$}
                    \end{prooftree}
                    $$\therefore\emptyset\vdash\varphi\lor\neg\varphi $$ 

                \item $ $
                    \begin{prooftree}
                        \AxiomC{$\mathfrak{A}\vDash\varphi$}
                        \UnaryInfC{Não-não $\mathfrak{A}\vDash\varphi$}
                        \UnaryInfC{Não $\mathfrak{A}\vDash\neg\varphi$}
                        \UnaryInfC{$\mathfrak{A}\vDash\neg\neg\varphi$}
                    \end{prooftree}
                    $$\therefore\varphi\vdash\neg\neg\varphi $$ 

                \item $ $
                    \begin{prooftree}
                        \AxiomC{$\mathfrak{A}\vDash\neg\neg\varphi$}
                        \UnaryInfC{Não $\mathfrak{A}\vDash\neg\varphi$}
                        \UnaryInfC{Não-não $\mathfrak{A}\vDash\varphi$}
                        \UnaryInfC{$\mathfrak{A}\vDash\varphi$}
                    \end{prooftree}
                    $$\therefore\neg\neg\varphi\vdash\varphi $$ 

                \item $ $
                    \begin{prooftree}
                        \AxiomC{${\varphi}\vdash\psi$}
                        \UnaryInfC{$\mathfrak{A}\vDash{\varphi}\RAR\mathfrak{A}\vDash\psi$}
                        \UnaryInfC{Não $\mathfrak{A}\vDash\varphi$ ou $\mathfrak{A}\vDash\psi$}
                        \UnaryInfC{$\mathfrak{A}\vDash\neg\varphi$ ou $\mathfrak{A}\vDash\psi$}
                        \UnaryInfC{$\mathfrak{A}\vDash\neg\varphi\lor\psi$}
                        \AxiomC{$   \mathfrak{A}\vDash\emptyset$}
                        \BinaryInfC{$\mathfrak{A}\vDash\emptyset\RAR\mathfrak{A}\vDash\varphi\rar\psi$}
                        \UnaryInfC{$\emptyset\vdash\varphi\rar\psi$}
                    \end{prooftree}

                    \begin{prooftree}
                        \AxiomC{$\emptyset\vdash\varphi\rar\psi$}
                        \UnaryInfC{$\mathfrak{A}\vDash\emptyset\RAR\mathfrak{A}\vDash\varphi\rar\psi$}
                        \UnaryInfC{$\mathfrak{A}\vDash\varphi\rar\psi$}
                        \UnaryInfC{Não $\mathfrak{A}\vDash\varphi$ ou $\mathfrak{A}\vDash\psi$}
                        \UnaryInfC{$\mathfrak{A}\vDash\varphi\RAR\mathfrak{A}\vDash\psi$}
                        %\UnaryInfC{$\mathfrak{A}\vDash\SET{\varphi}$\RAR$\mathfrak{A}\vDash\psi$}
                        \UnaryInfC{${\varphi}\vdash\psi$}
                    \end{prooftree}
                    $$\therefore {\varphi}\vdash\psi\BIM\emptyset\vdash\varphi\rar\psi $$

                \item $ $
                    \begin{prooftree}
                        \AxiomC{$\mathfrak{A}\vDash\forall x:\varphi$}
                        \UnaryInfC{Para todo $\hat x$ de $|\mathfrak{A}|$: $\mathfrak{A}\vDash_{f[x\slash{\hat x}]} \varphi$}
                        \AxiomC{$|\mathfrak{A}|\not=\emptyset$}
                        \UnaryInfC{Existe $\hat x$ em $|\mathfrak{A}|$}
                        \BinaryInfC{Existe $\hat x$ em $|\mathfrak{A}|$: $\mathfrak{A}\vDash_{f[x\slash{\hat x}]} \varphi$}
                        \UnaryInfC{$\mathfrak{A}\vDash\exists x:\varphi$}
                    \end{prooftree}
                    $$\therefore \forall x:\varphi\vdash\exists x:\varphi $$
                    
                \item $ $
                    \begin{prooftree}
                        \AxiomC{$\mathfrak{A}\vDash\forall x:\varphi$}
                        \UnaryInfC{Para todo $\hat x$ em $|\mathfrak{A}|$: $\mathfrak{A}_{f[x\slash\hat x]}\vDash\varphi$}
                        
                        \AxiomC{$\mathfrak{A}\vDash\exists x:\neg\varphi$}
                        \UnaryInfC{Existe $\hat x$ em $|\mathfrak{A}|$: $\mathfrak{A}_{f[x\slash\hat x]}\vDash\neg\varphi$}
                        
                        \BinaryInfC{$\bot$}
                    \end{prooftree}
                    $$\therefore \forall x:\varphi\vdash\neg\exists x:\neg\varphi $$
                \item $ $
                    $$\therefore \exists x:\neg\varphi\vdash\neg\forall x:\varphi $$
                \item Se $x$ não ocorre em $\varphi$, então é trivial. Se $x$ ocorre em 
                $\varphi$, lembramos que é uma sentença, então $x$ sempre ocorre 
                quantificado em $\varphi$. Então $\vDash$ vai ignorar os quantificadores 
                em ``$\forall x:\phi$'' e ``$\exists x:\phi$''.
                
                \item $ $
                    \begin{prooftree}
                        \AxiomC{$\mathfrak{A}\vDash\varphi$}
                        \AxiomC{$\mathfrak{A}\vDash\varphi\rar\psi$}
                        \UnaryInfC{$\mathfrak{A}\not\vDash\varphi$ ou $\mathfrak{A}\vDash\psi$}
                        \BinaryInfC{$\mathfrak{A}\vDash\psi$}
                    \end{prooftree}
                    $$\therefore\SET{\varphi}{\varphi\rar\psi}\vdash\psi $$
                
                \item $ $
                    \begin{prooftree}
                        \AxiomC{$\mathfrak{A}\vDash\neg\varphi$}
                        \UnaryInfC{Não $\mathfrak{A}\vDash\varphi$}

                        \AxiomC{$\mathfrak{A}\vDash\varphi\lor\psi$}
                        \UnaryInfC{$\mathfrak{A}\vDash\varphi$ ou $\mathfrak{A}\vDash\psi$}
                        
                        \BinaryInfC{$\mathfrak{A}\vDash\psi$}
                    \end{prooftree}
                    $$\therefore\SET{\varphi\lor\psi}{\neg\varphi}\vdash\psi $$
            \end{enumerate}
            \eop
        \end{proof}
        \paragraph{}
            O que estes resultados primários nos dizem é que, se tivermos uma prova formal 
            (em um sistema dedutivo razoável)
            --- partindo de hipóteses $\Gamma$ --- de uma coleção $\Phi$ de sentenças, 
            então temos garantido que $\Gamma\vdash\Phi$. Isto é importante porque temos 
            que, de certa forma, $\vdash$ respeita a dedução lógica: se achamos que 
            $\Gamma$ consegue provar $\Phi$, então de fato onde vale $\Gamma$, vale $\Phi$.
        \paragraph{}
            A reciproca, {\emph{que toda sentênça consequente de $\Gamma$ é provável por 
            hipóteses de $\Gamma$}}, requer um trato cuidadoso com sistemas dedutíveis, 
            definição de prova, {\emph{etc.}} Mas é resultado conhecido que se $\Gamma\vdash\varphi$,
            então {\emph{prova-se}} $\varphi$ com hipóteses de $\Gamma$, mas isso depende do 
            sistema dedutivo.