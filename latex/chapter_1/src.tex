\chapter{Lógica}
    \epigraph{\justify
        It seems to me now that mathematics 
        is capable of an artistic excellence 
        as great as that of any music 
        \elide because it gives in absolute 
        perfection that combination, 
        characteristic of great art, of godlike 
        freedom, with the sense of inevitable 
        destiny; because, in fact, it constructs 
        an ideal world where everything is 
        perfect and yet true.
        }{\textit{Bertrand Russell.}}    
    \cls
    \paragraph{}
        A forma mais básica de lógica que nos propomos
        a tratar é a ``lógica proposicional'', que 
        veremos ser intimamente ligada a álgebras de
        Heyting. Ela consiste de uma coleção de letras
        proposicionais, que podem ser vistas como 
        condições, e destas ``fórmulas atômicas'', 
        obtemos fórmulas compostas através do uso de 
        conectivos e operadores lógicos.
    \paragraph{}
        O estudo da lógica proposicional tem seu 
        mérito, mas não será especialmente util para
        nós. O nosso foco serão álgebras booleanas 
        completas. Veremos, no entanto, um tanto sobre
        álgebras de heyting, que são como que realizações 
        de alguma lógica proposicional.
    \paragraph{}
        Grosseiramente, uma lógica proposicional se ocupa
        de sentenças como:
        $$(P\rar Q)\rar P\rar Q$$
        $$(P\land Q)\rar P$$
    \paragraph{}
        Onde $P$, $Q$, {\emph etc.} são letras 
        proposicionais. Uma álgebra de Heyting 
        como que dá um valor parcialmente ordenado
        a cada uma das letras, e indutivamente, 
        obtemos um valor para fórmulas como as 
        acima.

    \section{Lógica de Primeira Ordem}
        \paragraph{}
            Definimos uma língua formal como sendo 
            as sequências simbólicas finitas definidas
            por pela seguinte gramática:
        \paragraph{}
            Dados $\Sigma=\SET{R_0}{\ldots}{R_n}$ símbolos 
            relacionais $\rho_0,\ldots,\rho_n$-árias,
            respectivamente; e uma coleção $\Omega
            =\SET{x_0}{x_1}{x_2}{\ldots}$ de símbolos 
            de variáveis (infinitos e distintos dos relacionais)
            que uma sequência (finita) de símbolos $\sigma$ 
            pertence à língua de $\TUPLE{\Sigma}{\Omega}$  
            exatamente quando vale alguma das 
            condições abaixo:
        \begin{enumerate}[label=\alph*)]
            \item $\sigma\equiv R_k\POINT{x_{\alpha_0}}{\ldots}{x_{\alpha_{\rho_k}}}$, com $0\leq k\leq n$, e $x_{\alpha_i}$ todos em $\Omega$.
            \item $\sigma\equiv \varphi\land\psi$, com $\varphi,\psi$ da língua.
            \item $\sigma\equiv \varphi\lor \psi$, com $\varphi,\psi$ da língua.
            \item $\sigma\equiv \neg\varphi$, com $\varphi$ da língua.
            \item $\sigma\equiv \forall x_i:\varphi$, com $x_i$ variável em $\Omega$ e $\varphi$ da língua.
            \item $\sigma\equiv \exists x_i:\varphi$, com $x_i$ variável em $\Omega$ e $\varphi$ da língua.
        \end{enumerate}
        \paragraph{}
            Como na construção da língua, usamos 
            apenas a aridade e número de 
            símbolos relacionais, podemos 
            definir uma língua canônica para 
            cada ``assinatura''. 
        \paragraph{}
            Primeiro, fixaremos uma coleção 
            de símbolos $\Omega=\SET{x_0}
            {x_1}{\ldots}$, para variáveis, 
            e uma coleção de símbolos 
            distintos $\Sigma=\SET{R_0}
            {R_1}{\ldots}$ relacionais.
            Para cada $\rho:n\longrightarrow\N$
            que define a aridade de cada um dos 
            $R_i$-s e número de símbolos relacionais,
            define-se a língua $\mathcal{L}_\rho$ 
            como feito acima.
            

        \subsection{Satisfatibilidade}
        \paragraph{}
            Sem alguma ideia de o que fórmulas de 
            uma língua dada dizem a respeito de um 
            ambiente, não temos nada de muito interessante.
            Então, vamos definir o que queremos dizer com 
            um ambiente e, então definir o que significa 
            satisfatibilidade. 
        \paragraph{}
            Primeiro, para definirmos \textbf{estruturas}, 
            pois se uma fórmula é verdade ela deve
            ser verdade em algum lugar. Para tanto, 
            vamos definir uma 
            {\textbf{valoração de variáveis}}:

        \begin{definition}{Valoração de Variáveis}
                Dada uma classe $A$ de objetos, uma
                valoração de 
                \textbf{variáveis das variáveis} 
                de uma língua $\mathcal{L}$ de 
                primeira ordem com alguma assinatura,
                dizemos que $f$ é uma \textbf{valoração de 
                variáveis} quando ela é uma realização
                dos símbolos das $\mathcal{L}$-variáveis
                em objetos de $A$ --- {\emph ie.} uma função 
                que associa símbolos de variáveis a objetos 
                concretos de $A$.
            \paragraph{}
                Adicionalmente, dados um objeto $o$ 
                de $A$, uma variável $x_j$ de 
                $\mathcal{L}$ e uma valoração $f$, 
                definimos:
                $$f[x_j\slash{o}](x_i) = \casedef{
                    o\text{, &se }i = j;\\
                    f(x_i)\text{, &se }i\not=j.
                }$$ 
            \paragraph{}
                Uma valoração que é a substituição do 
                valor de $f$ em um só ponto. 
        \end{definition}

        \paragraph{}
            Uma \textbf{estrutura} é uma coleção de informações:
            $$\mathfrak{A} = \TUPLE{A}{R_0}{\ldots}{R_n}$$
        \paragraph{}
            Onde $A$ é uma classe não vazia chamada ``domínio'' 
            e cada uma $R_i$ é uma relação definida 
            sobre o mesmo. Dizemos que uma estrutura 
            e uma língua são \textbf{compatíveis}
            exatamente quando a aridade de cada $R_i$ 
            é $\rho_i$.

        \begin{definition}{Satisfatibilidade}
                Sejam $\varphi$ uma $\mathcal{L}$-fórmula;  
                $\mathfrak{A} = \TUPLE{A}{P_0}{\ldots}{P_n}$
                uma estrutura;
                $f$ uma valoração de variáveis em $A$; e 
                sejam $\mathcal{L}$ e $\mathfrak{A}$ compatíveis.
                $$ \mathfrak A\vDash_f\varphi$$
            \paragraph{}
                Que se lê  ``$\mathfrak{A}$ satisfaz 
                $\varphi$'', ``$\mathfrak{A}$ crê que 
                $\varphi$'' ou ``$\varphi$ vale em 
                $\mathfrak{A}$'', É definido recursivamente:
            \begin{align*}
                \varphi\equiv R_i\POINT{x_0}{\ldots}{x_{\alpha_{\rho_i}}}
                    &\RAR \mathfrak A\vDash_f\varphi \bim P_i\POINT{f(x_0)}{\ldots}{f(x_{\alpha_{\rho_i}})}.\\
                \varphi\equiv \sigma\land\psi    
                    &\RAR \mathfrak A\vDash_f\varphi \bim \mathfrak A\vDash_f\sigma \text{ e } \mathfrak A\vDash_f\psi.\\
                \varphi\equiv \sigma\lor \psi    
                    &\RAR \mathfrak A\vDash_f\varphi \bim \mathfrak A\vDash_f\sigma \text{ ou } \mathfrak A\vDash_f\psi.\\
                \varphi\equiv \neg\sigma         
                    &\RAR \mathfrak A\vDash_f\varphi \bim \text{ não }\mathfrak A\vDash_f\sigma.\\
                \varphi\equiv \sigma\rar\psi         
                    &\RAR \mathfrak A\vDash_f\varphi \bim \mathfrak A\vDash_f\neg\sigma \text{ ou } \mathfrak A\vDash_f\psi.\\
                \varphi\equiv \forall x_i:\sigma 
                    &\RAR \mathfrak A\vDash_f\varphi \bim \text{ para todo $a$ de $A$: } \mathfrak A\vDash_{f[x_i\slash{a}]}\varphi.\\
                \varphi\equiv \exists x_i:\sigma 
                    &\RAR \mathfrak A\vDash_f\varphi \bim \text{ existe algum $a$ de $A$: } \mathfrak A\vDash_{f[x_i\slash{a}]}\varphi.
            \end{align*}
            \paragraph{}
                Existe uma questão delicada nessa quantificação
                que fazemos nas últimas duas clausulas: Não devemos
                acreditar que é possível fazer isso em $1^a$ ordem,
                devido a um resultado de Tarski. Uma língua que pode 
                definir satisfatibilidade para todas as outras línguas 
                de primeira ordem, pode, também definir para ela mesma,
                e isso resulta em uma contradição devido a meta-linguística.
        \end{definition}

        \paragraph{}
            Vamos tomar um parágrafo para explorar o significado 
            da definição (admitidamente meta-teorética) de 
            satisfatibilidade. Primeiro, vemos que se a fórmula 
            sendo interpretada é um sentença\footnote{{\emph ie.} 
            todas as variáveis da fórmula estão quantificadas 
            sobre por algum $\forall$ ou um $\exists$, em todos 
            os escopos.}, então não importa quais valorações $f,
            f'$ que dermos, dada uma $\sigma$-sentença:
            $$ \mathfrak{A}\vDash_{f} \sigma \BIM \mathfrak{A}\vDash_{f'} \sigma $$
        \paragraph{}
            Segundo, se uma fórmula da língua é $\varphi\POINT{x}{y}{z}$\footnote{
            Onde $x,y,z$ são açucar sintático para alguns $x_i, 
            x_j, x_k$.} onde estas vairáveis estão livres na 
            fórmula, então se $\hat{x}, \hat{y}, \hat{z}$ forem
            objetos do domínio de uma estrutura compatível, 
            ``$\varphi\POINT{\hat{x}}{\hat{y}}{\hat{z}}$'' ---
            que gostaríamos que significasse quão verdade é dizer que
            $\hat{x}, \hat{y}, \hat{z}$ estão relacionados como
            $\varphi$ predica --- se traduz simplesmente em:
            $f'=f[x\slash\hat x; y\slash\hat y; z\slash\hat z]$
            uma nova valoração de variáveis:
            $$ \mathfrak{A}\vDash_{f'} \varphi $$
        \paragraph{}
            Codifica exatamente isso. $\vDash_{f'}$ vale se e só
            se a estrutura dá uma semiose à $\varphi$ ``avaliada''
            nos objetos $\hat{x},\hat{y},\hat{z}$. Assim, 
            satisfatibilidade é a codificação dessa interpretação 
            de uma coisa {\emph sintática}, a fórmula, em uma coisa 
            {\emph semântica} relações entre objetos, quantificações 
            sobre o domínio, etc.
        \paragraph{}
            Finalmente, --- dado um coleção de fórmulas $\Gamma$\footnote{
                Caso $\Gamma$ seja vazio, $\mathfrak{A}\vDash\Gamma$ é vacuamente verdade.
            }
            de uma língua compatível com $\mathfrak A$ estrutura e uma 
            valoração $f$ ---, dizemos que:
            $$\mathfrak{A}\vDash_f\Gamma\BIM\text{ para toda }\varphi\text{ de }\Gamma\text{: }\mathfrak{A}\vDash_f\varphi$$
        \paragraph{}
            E que, se $\Gamma$ for coleção de sentenças, então dizemos:
            $$\mathfrak{A}\vDash\Gamma\BIM\text{ para toda }\varphi\text{ de }\Gamma\text{: }\mathfrak{A}\vDash_f\varphi$$
            onde $f$ é qualquer (fixa) valoração das variáveis, por exemplo: $f(x_i)=\hat{x}$\footnote{Pelo menos uma tal 
            valoração existe pois os domínios de estruturas são não vazios.}

        \subsection{Provabilidade}
        \paragraph{}
            O interessante da satisfatibilidade como 
            definida antes é sua relação com a
            provabilidade. Para explora-la, devemos 
            tratar de ``teorias''.
        \paragraph{}
            Agora, sendo um pouco colecionista, 
            considere o conjunto das fórmulas de 
            de uma língua $\mathcal{L}_\Sigma$,
            com $\Sigma = \TUPLE{R_0}{\ldots}{R_n}$.
        \begin{definition}{Consequência Sintática}
                A ideia por trás da definição é que existem coisas 
                que são verdadeiras independente de interpretação e 
                localidade. Por exemplo, não importa se 
                $\forall x_7\forall x_8:R_3(x_7,x_8)$ é verdade ou não
                em um uma dada estrutura (de assinatura compatível), 
                sempre vale que: $[\forall x_7\forall x_8:R_3(x_7,x_8)]
                \rar[\exists x_7:\exists x_8:R_3(x_7,x_8)]$. Isto é, 
                existem sentenças que não dependem de interpretações de 
                variáveis nem da semântica das relações: suas verdades 
                são {\emph consequências sintáticas}.
            \paragraph{}
                Então vamos definir uma relação entre conjuntos de sentenças 
                de uma língua e sentenças desta mesma língua a de consequência
                sintática.
            \paragraph{}
                Seja $\Gamma\subset\mathcal{L}$ conjunto de sentenças de uma 
                língua. Seja também $\varphi$ uma sentença da mesma língua. 
                Dizemos que $\varphi$ é consequência sintática de $\Gamma$:
                $$\Gamma\vdash\varphi\BIM(\mathfrak{A}\vDash\Gamma\RAR\mathfrak{A}\vDash\varphi)\footnote{
                    Onde $\mathfrak{A}$ é uma estrutura de assinatura compatível. Estamos, de certa forma,
                    com $\mathfrak{A}$ livre sobre estruturas, o que é assustador. Afinal, para teoria de 
                    conjuntos, por exemplo, potencialmente estaríamos quantificando sobre classe próprias. 
                    E, em teorias mais fortes, todo tipo de coisas informalizáveis. Já estamos, no entanto, 
                    em uma meta-teoria generosa.\\ Pouco mais abaixo de onde estamos se agita um mar de metafísica, e questões 
                    epistemo e ontológicas no qual não ousamos mergulhar.
                }$$
        \end{definition}
        \paragraph{}
            A definição se extende para conjuntos de sentenças simplemente dizendo
            $$\Gamma\vdash\Sigma\BIM\text{ para todo $\sigma$ de $\Sigma$ temos: } \Gamma\vdash\sigma$$
        \paragraph{}
            Em se tratando de conjuntos unitários, {\emph ie.} $\Gamma=\SET{\varphi}$, abreviamos a relação $\vdash$ para:
            $$\varphi\vdash\psi\BIM\SET{\varphi}\vdash\psi$$
        \paragraph{}
            Esta é uma definição de fato bem estranha, definimos uma coisa ser sintáticamente 
            consequente da outra em termos de interpretações, que são semânticas em 
            natureza\footnote{
                É claro que é possível fazê-lo de maneira puramente sintática, que é útil 
                para computadores e provadores automáticos de teoremas, mas cremos que é 
                uma abordagem que oculta algo da beleza da relação inseparável entre $\vdash$ 
                e $\vDash$.\\
                Por outro lado, uma abordagem sintática {\emph é} de fato mais precisa, pois não
                necessita quantificar sobre as estruturas compatíveis com a língua, 
                {\emph etc.}
            }. No entanto,
            ela é uma relação extremamente razoável: não há interpretação que tenha $\Gamma$ 
            sem ter $\varphi$. De alguma forma, dizer que ``Em todo lugar que $\Gamma$ vale, 
            $\varphi$ vale'' nos garante que não importa o lugar, apenas $\Gamma$ e $\varphi$.
        
        \subsubsection{Teorias}
        \begin{definition}{Teoria}
                Dada uma língua, um subconjunto de sentenças $\mathcal{T}$ é dito uma teoria
                quando ele é não-vazio e $\vdash$-fechado. Isto é, se $\mathcal{T}\vdash\tau$ 
                então $\tau$ já estava em $\mathcal{T}$. Ou seja, é uma coleção de sentenças 
                que contém todas as suas consequências semânticas.
        \end{definition}
        \paragraph{}
            \newcommand{\theoryof}[1]{\mathcal{T}_{#1}}
            Dada uma língua $\mathcal{L}$, e um conjunto de sentenças não-vazio desta que batizamos ``axiomas'' 
            $\mathcal{A}$, dizemos que um conjunto de sentenças $\mathcal{T}$ é a teoria de 
            $\mathcal{A}$ exatamente quando {\emph para toda $\tau$ de $\mathcal{T}$, temos 
            $\mathcal{A}\vdash\tau$}, que podemos abreviar para $\theoryof{\mathcal{A}}$.
        \paragraph{}
            O fato de que, para um conjunto de axiomas conforme acima, $\theoryof{\mathcal{A}}$ é teoria 
            verifica-se por:
        \begin{proof*}
            \begin{prooftree}
                \AxiomC{$\theoryof{\mathcal{A}}\vdash\tau$}
                \UnaryInfC{$\mathfrak{A}\vDash\theoryof{\mathcal{A}}\RAR\mathfrak{A}\vDash\tau$}

                \AxiomC{$\mathcal{A}\subseteq\theoryof{\mathcal{A}}$}
                \UnaryInfC{$\mathfrak{A}\vDash\theoryof{\mathcal{A}}\RAR\mathfrak{A}\vDash\mathcal{A}$}

                \AxiomC{para cada $\sigma$ em $\theoryof{\mathcal{A}}$}
                \UnaryInfC{$\mathcal{A}\vdash\sigma$}
                \UnaryInfC{$\mathfrak{A}\vDash\mathcal{A}\RAR\mathfrak{A}\vDash\theoryof{\mathcal{A}}$}
                
                \BinaryInfC{$\mathfrak{A}\vDash\mathcal{A}\BIM\mathfrak{A}\vDash\theoryof{\mathcal{A}}$}
                \BinaryInfC{$\mathfrak{A}\vDash\mathcal{A}\RAR\mathfrak{A}\vDash\tau$}

                \UnaryInfC{$\mathcal{A}\vdash\tau$}
                \UnaryInfC{$\tau$ está em $\theoryof{\mathcal{A}}$}
            \end{prooftree}
            \eop
        \end{proof*}
        
        \begin{proposition}{Resultados sobre $\vdash$}
            Dada uma língua fixada $\mathcal{L}$ e 
            $\varphi,\psi,\sigma,\tau$ sentenças da
            de $\mathcal{L}$.
            \label{vdash_results_00}
            \begin{enumerate}[label=\alph*)]
                \item $    {\varphi}\vdash\varphi$,
                \item $\Gamma\vdash\varphi$ e $\Gamma\subseteq\Lambda\RAR \Lambda\vdash\varphi$,
                \item $    {\varphi}\vdash\varphi\lor\psi$,
                \item $\SET{\varphi}{\psi}\vdash\varphi\land\psi$,
                \item $    {\varphi\land\psi}\vdash\SET{\varphi}{\psi}$,
                \item $\emptyset\vdash  \varphi\lor\neg\varphi$,
                \item $\varphi\vdash\neg\neg\varphi$,
                \item $\neg\neg\varphi\vdash\varphi$,
                \item $    {\varphi}\vdash\psi\BIM\emptyset\vdash\varphi\rar\psi$,
                \item $    {\forall x:\varphi}\vdash\exists x:\varphi$,
                \item $    {\forall x:\varphi}\vdash\neg\exists x:\neg\varphi$,
                \item $    {\neg\exists x:\neg\varphi}\vdash\forall x:\varphi$,
                \item $\emptyset\vdash\varphi\RAR\emptyset\vdash\SET{\forall x:\varphi}{\exists x:\varphi}$
            \end{enumerate}
        \end{proposition}
        \begin{proof}
            \begin{enumerate}[label=\alph*)]
                \item $ $
                    \begin{prooftree}
                        \AxiomC{$\mathfrak{A}\vDash{\varphi}$}
                        \UnaryInfC{$\mathfrak{A}\vDash\varphi$}
                        \UnaryInfC{$\mathfrak{A}\vDash{\varphi}\RAR\mathfrak{A}\vDash\varphi$}
                        \UnaryInfC{${\varphi}\vdash\varphi$}
                    \end{prooftree}
                \item $ $
                    \begin{prooftree}
                        \AxiomC{$\mathfrak{A}\vDash\Lambda$}

                        \AxiomC{$\Gamma\vdash\varphi$}
                        \UnaryInfC{$\mathfrak{B}\vDash\Gamma\RAR\mathfrak{B}\vDash\varphi$}

                        \AxiomC{$\Gamma\subseteq\Lambda$}
                        
                        \AxiomC{$\mathfrak{A}\vDash\Lambda$}
                        \UnaryInfC{$\lambda\in\Lambda\RAR\mathfrak{A}\vDash\lambda$}
                        
                        \BinaryInfC{$\lambda\in\Gamma \RAR\mathfrak{A}\vDash\lambda$}
                        \UnaryInfC{$\mathfrak{A}\vDash\Gamma$}
                    
                        \BinaryInfC{$\mathfrak{A}\vDash\varphi$}
                    \end{prooftree}
                    $$\therefore\mathfrak{A}\vDash\Lambda\RAR\mathfrak{A}\vDash\varphi$$
                    $$\text{Assim, } \Lambda\vdash\varphi$$

                \item $ $
                    \begin{prooftree}
                        \AxiomC{   $\mathfrak{A}\vDash{\varphi}$}
                        \UnaryInfC{$\mathfrak{A}\vDash{\varphi}$ ou $\mathfrak{A}\vDash{\psi}$}
                        \UnaryInfC{$\mathfrak{A}\vDash{\varphi\lor\psi}$}
                    \end{prooftree}                                            
                    $$\therefore{\varphi}\vdash\varphi\lor\psi $$

                \item $ $
                    \begin{prooftree}
                        \AxiomC{$\mathfrak{A}\vDash\SET{\varphi}{\psi}$}
                        \UnaryInfC{$\mathfrak{A}\vDash\varphi$}
                        \AxiomC{$\mathfrak{A}\vDash\SET{\varphi}{\psi}$}
                        \UnaryInfC{$\mathfrak{A}\vDash\psi$}
                        \BinaryInfC{$\mathfrak{A}\vDash\varphi$ e $\mathfrak{A}\vDash\psi$}
                        \UnaryInfC{$\mathfrak{A}\vDash\varphi\land\psi$}
                    \end{prooftree}
                    $$\therefore\SET{\varphi}{\psi}\vdash\varphi\land\psi $$
                
                \item $ $
                    \begin{prooftree}
                        \AxiomC{$\mathfrak{A}\vDash{\varphi\land\psi}$}
                        \UnaryInfC{$\mathfrak{A}\vDash{\varphi}$ e $\mathfrak{A}\vDash{\psi}$}
                        \UnaryInfC{$\mathfrak{A}\vDash\SET{\varphi}{\psi}$}
                    \end{prooftree}
                    $$\therefore{\varphi\land\psi}\vdash\SET{\varphi}{\psi} $$
                
                \item $ $
                    \begin{prooftree}
                        \AxiomC{$\mathfrak{A}\vDash\emptyset$}
                        \AxiomC{$\mathfrak{A}\not\vDash\varphi$}
                        \BinaryInfC{Não $\mathfrak{A}\vDash\varphi$}
                        \UnaryInfC{$\mathfrak{A}\vDash\neg\varphi$}
                    \end{prooftree}
                    $$\therefore\emptyset\vdash\varphi\lor\neg\varphi $$ 

                \item $ $
                    \begin{prooftree}
                        \AxiomC{$\mathfrak{A}\vDash\varphi$}
                        \UnaryInfC{Não-não $\mathfrak{A}\vDash\varphi$}
                        \UnaryInfC{Não $\mathfrak{A}\vDash\neg\varphi$}
                        \UnaryInfC{$\mathfrak{A}\vDash\neg\neg\varphi$}
                    \end{prooftree}

                    $$\therefore\varphi\vdash\neg\neg\varphi $$ 

%%
                \item $ $
                    \begin{prooftree}
                        \AxiomC{$\mathfrak{A}\vDash\neg\neg\varphi$}
                        \UnaryInfC{Não $\mathfrak{A}\vDash\neg\varphi$}
                        \UnaryInfC{Não-não $\mathfrak{A}\vDash\varphi$}
                        \UnaryInfC{$\mathfrak{A}\vDash\varphi$}
                    \end{prooftree}

                    $$\therefore\neg\neg\varphi\vdash\varphi $$ 


                \item $ $
                    \begin{prooftree}
                        \AxiomC{${\varphi}\vdash\psi$}
                        \UnaryInfC{$\mathfrak{A}\vDash{\varphi}\RAR\mathfrak{A}\vDash\psi$}
                        \UnaryInfC{Não $\mathfrak{A}\vDash\varphi$ ou $\mathfrak{A}\vDash\psi$}
                        \UnaryInfC{$\mathfrak{A}\vDash\neg\varphi$ ou $\mathfrak{A}\vDash\psi$}
                        \UnaryInfC{$\mathfrak{A}\vDash\neg\varphi\lor\psi$}
                        \AxiomC{$   \mathfrak{A}\vDash\emptyset$}
                        \BinaryInfC{$\mathfrak{A}\vDash\emptyset\RAR\mathfrak{A}\vDash\varphi\rar\psi$}
                        \UnaryInfC{$\emptyset\vdash\varphi\rar\psi$}
                    \end{prooftree}

                    \begin{prooftree}
                        \AxiomC{$\emptyset\vdash\varphi\rar\psi$}
                        \UnaryInfC{$\mathfrak{A}\vDash\emptyset\RAR\mathfrak{A}\vDash\varphi\rar\psi$}
                        \UnaryInfC{$\mathfrak{A}\vDash\varphi\rar\psi$}
                        \UnaryInfC{Não $\mathfrak{A}\vDash\varphi$ ou $\mathfrak{A}\vDash\psi$}
                        \UnaryInfC{$\mathfrak{A}\vDash\varphi$\RAR$\mathfrak{A}\vDash\psi$}
                        %\UnaryInfC{$\mathfrak{A}\vDash\SET{\varphi}$\RAR$\mathfrak{A}\vDash\psi$}
                        \UnaryInfC{${\varphi}\vdash\psi$}
                    \end{prooftree}
                    $$\therefore {\varphi}\vdash\psi\BIM\emptyset\vdash\varphi\rar\psi $$

                \item $ $
                    \begin{prooftree}
                        \AxiomC{$\mathfrak{A}\vDash\forall x:\varphi$}
                        \UnaryInfC{Para todo $\hat x$ de $|\mathfrak{A}|$: $\mathfrak{A}\vDash_{f[x\slash{\hat x}]} \varphi$}
                        \AxiomC{$|\mathfrak{A}|\not=\emptyset$}
                        \UnaryInfC{Existe $\hat x$ em $|\mathfrak{A}|$}
                        \BinaryInfC{Existe $\hat x$ em $|\mathfrak{A}|$: $\mathfrak{A}\vDash_{f[x\slash{\hat x}]} \varphi$}
                        \UnaryInfC{$\mathfrak{A}\vDash\exists x:\varphi$}
                    \end{prooftree}
                    $$\therefore \forall x:\varphi\vdash\exists x:\varphi $$
                    
                \item $ $
                    \begin{prooftree}
                        \AxiomC{$\mathfrak{A}\vDash\forall x:\varphi$}
                        \UnaryInfC{Para todo $\hat x$ em $|\mathfrak{A}|$: $\mathfrak{A}_{f[x\slash\hat x]}\vDash\varphi$}
                        
                        \AxiomC{$\mathfrak{A}\vDash\exists x:\neg\varphi$}
                        \UnaryInfC{Existe $\hat x$ em $|\mathfrak{A}|$: $\mathfrak{A}_{f[x\slash\hat x]}\vDash\neg\varphi$}
                        
                        \BinaryInfC{$\bot$}
                    \end{prooftree}
                    $$\therefore \forall x:\varphi\vdash\neg\exists x:\neg\varphi $$
                \item $ $
                    $$\therefore \exists x:\neg\varphi\vdash\neg\forall x:\varphi $$
                \item Se $x$ não ocorre em $\varphi$, então é trivial. Se $x$ ocorre em 
                $\varphi$, lembramos que é uma sentença, então $x$ sempre ocorre 
                quantificado em $\varphi$. Então $\vDash$ vai ignorar os quantificadores 
                em ``$\forall x:\phi$'' e ``$\exists x:\phi$''.
            \end{enumerate}
            \eop
        \end{proof}
        \paragraph{}
            O que estes resultados primários nos dizem é que, se tivermos uma prova formal 
            --- partindo de hipóteses $\Gamma$ --- de uma coleção $\Phi$ de sentenças, 
            então temos garantido que $\Gamma\vdash\Phi$. Isto é importante porque temos 
            que, de certa forma, $\vdash$ respeita a dedução lógica: se achamos que 
            $\Gamma$ consegue provar $\Phi$, então de fato onde vale $\Gamma$, vale $\Phi$.
        \paragraph{}
            O importante agora é ver se ``Se $\Gamma\vdash\varphi$, então existe uma 
            \textbf{prova} de $\varphi$ partindo de $\Gamma$''. Ou seja, que consequência 
            sintática é equivalente a provabilidade.