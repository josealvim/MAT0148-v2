\chapter{$V^{(B)}$}
%    \epigraph{\justify
%            A successful attempt to express 
%            logical propositions by symbols, 
%            the laws of whose combinations 
%            should be founded upon the laws 
%            of the mental processes which 
%            they represent, would, so far, 
%            be a step towards a philosophical 
%            language.
%        }{\textit{George Boole -- The Mathematical Analysis of Logic. p.5}}    
    \cls
    \paragraph{}
        Queremos modelos diferentes de $\ZF$
        a fim de encontrar exemplos de modelos 
        com propriedades diferentes de $L$, 
        por exemplo, onde não valha o Axioma 
        da Escolha, ou a Hipótese do Contínuo.
    \begin{definition}{$ V^{(B)} $}
        Seja $B$ uma Álgebra Booleana Completa
        $$ V^{(B)}_{\alpha} = \SET{ f : \exists\beta<\alpha:\exists D\subseteq V^{(B)}_{\beta} : f\in D^B } $$
        $$ V^{(B)} = \SET{f : \exists\alpha: f\in V^{(B)}_{\alpha}} $$
    \end{definition}
    \paragraph{}
        Esta classe que definimos acima é 
        uma espécie de homogenização da 
        ideia de função característica. No 
        sentido de que a função $\chi$ 
        leva um conjunto em um representante
        de determinada classe de funções: as 
        funções características daquele 
        conjunto. O que estamos fazendo em 
        $V^{(B)}$ é criar um tipo 
        estratificado que, cujos membros são 
        funções dos membros anteriores em uma 
        álgebra booleana completa dada.
    \paragraph{}
        A maneira de pensar sobre esta classe
        é que seus membros são representantes 
        de objetos --- que podem até não estar
        no modelo base --- cujas propriedades 
        queremos capturar. Cada representante
        decide \textit{quanto} o representado 
        por membros anteriores \textit{pertence}
        ao conjunto que ele representa em um
        determinado modelo diferente.
    \paragraph{}
        Trivialmente $\alpha<\beta\bim\V{B}_\alpha\subset\V{B}_\beta$
    \begin{definition}{$Sat_{\V{B}}$}
            Vamos emprestar a notação de Drake,
        $$ Sat(\V{B}, \varphi, f) $$ 
        \paragraph{}
            Passa a ser uma \textit{função}:
        $$ Sat_{\V{B}}:\Phi\times{\V{B}}^{<\omega} \longrightarrow B$$
        Definida por recursão na complexidade das fórmulas:
        \paragraph{}
            Se $B = \TUPLE{|B|}{\lor}{\land}{\neg}{1}{0}$, então
        \begin{align*}
            Sat(\V{B}, \FMLA{\varphi\land\psi}, f) &=      Sat(\V{B}, \FMLA{\varphi},f) \land Sat(\V{B}, \FMLA{\psi},f)\\
            Sat(\V{B}, \FMLA{\varphi\lor\psi},  f) &=      Sat(\V{B}, \FMLA{\varphi},f) \lor  Sat(\V{B}, \FMLA{\psi},f)\\
            Sat(\V{B}, \FMLA{\varphi\rar\psi},  f) &= \neg Sat(\V{B}, \FMLA{\varphi},f) \lor  Sat(\V{B}, \FMLA{\psi},f)\\
            Sat(\V{B}, \FMLA{\neg\psi},         f) &=                                   \neg  Sat(\V{B}, \FMLA{\psi},f)\\
            Sat(\V{B}, \FMLA{\forall x_i: \psi},f) &= \bigwedge_{x\in\V{B}}                   Sat(\V{B}, \psi, f_{[x_i\slash{x}]})\\
            Sat(\V{B}, \FMLA{\exists x_i: \psi},f) &= \bigvee_{  x\in\V{B}}                   Sat(\V{B}, \psi, f_{[x_i\slash{x}]})
        \end{align*}
        \paragraph{}
            Para as fórmulas atômicas temos um problema bem maior. 
            O que tentávamos fazer acima era emular a definição 
            de $\vDash$ para estruturas mas internamente. Entrentanto, 
            como $\V{B}$ não é nem um pouco transitivo, a definição 
            de $=$ e de $\in$ não pode ser a simples restrição das 
            relações em $V$.
        \paragraph{}
            Queremos, como dissemos, representar objetos de um 
            universo potencialmente diferente do universo base. 
            Mas, veremos, há muitos representantes redundantes,
            a relação de igualdade e de pertinência devem como 
            que quocientar estes representantes e simplesmente 
            identificar coisas que reconhecemos como diferentes
            no nosso universo. De qualquer forma, define-se por
            recursão:
        $$ Sat(\V{B}, \FMLA{x\in y}, f) = \bigvee_{\hat z\in Dom(f[y])} (f[y])(\hat z)\land Sat(\V{B}, \FMLA{x = z}, f_{[z\slash\hat z]} )$$
        \paragraph{}
            Onde $f[y]$ é o elemento valorado como $y$ por $f$. 
            Bom, ainda falta definir a igualdade, que usamos 
            para definir pertinência.
        \begin{align*}
            Sat(\V{B}, &\FMLA{x = y}, f) = \\
                &\left[\bigwedge_{\hat u \in Dom(f[x])} (f[x])(\hat u) \rar Sat(\V{B}, \FMLA{u \in y}, f_{[u\slash\hat u]}) \right] \land \\
                &\left[\bigwedge_{\hat v \in Dom(f[y])} (f[y])(\hat v) \rar Sat(\V{B}, \FMLA{v \in x}, f_{[v\slash\hat v]}) \right]
        \end{align*}
    \end{definition}
    \paragraph{}
        O complicado desta definição é a forma amarrada que 
        ela se dá: a recursão acontece de maneira estranha,
        e precisamos mostrar que a relação da recursão é bem
        fundanda. \cite{Bell} faz isto com algum detalhe, e,
        por ser apenas uma tecnicalidade e não ilustra nada 
        da teoria em si, não reporduziremos a curta prova de
        que o que definimos acima é de fato função e existe,
        antes de qualquer outra coisa.
    \paragraph{}
        Em termos de discussão que podemos fazer nesta etapa,
        há alguns pontos particularmente interessantes: 
        primeiramente, o motivo de exigirmos álgebras booleanas
        \textit{completas} é que estes \textit{sups} e 
        \textit{infs} precisam existir sempre.
    \paragraph{}
        Em segundo lugar, a definição de $Sat$, que funciona 
        muito bem para fórmulas compostas, e faz mais ou menos 
        o que se espera dela. Mas, nos casos de fórmulas 
        atômicas, age de maneira estranha mas familiar:
    \paragraph{}
        Dois objetos de, digamos, ``${\V{B}}\slash{\tilde=}$''
        são iguais quando seus representantes entram em uma 
        sorte de acordo: $\V{B}$ acredita que se um representante
        diz que um outro objeto pertence ao seu cliente, então 
        o objeto pertence ao cliente do outro. E \textit{vice 
        versa}. É fato que $\V{2}$ é essencialmente igual a $V$, 
        assim, o que dá sabor a esta mistura é, realmente, $B$,
        que permite que coisas ``mais ou menos'' pertençam a 
        outras, \textit{etc}.
    \paragraph{}
        Uma discussão importante também é a dos morfismos de 
        álgebras booleanas completas. Em parte por conta de 
        propriedades futuras e importantes, mas também para 
        nos familiarizarmos com a estrutura.

    \begin{proposition}{Morfismos de Álgebras induzem morfismos entre $\V{B}$s}
        Sejam $A$ e $B$ duas álgebras booleanas completas e seja 
        $$\omega: A\longrightarrow B$$
        Um morfismo de álgebras booleanas que preserva \textit{sups} arbitrários.
        Define-se indutivamente uma série de funções compatíveis $\Omega_\alpha$:
        Suponha que, dado um $\alpha$
        $$[F:\bigcup_{\beta<\alpha}\V{A}_\beta\longrightarrow\V{B}] \land [F(x) = \varphi\circ x\circ F]$$
        Então deixe 
        \begin{align*}
            \Omega_\alpha : &\V{A}_\alpha \longrightarrow \V{B}\\
                            & x           \longmapsto \begin{cases}
                F(x),                 \text{ caso } \exists\beta<\alpha:x\in\V{A}_\beta\\
                \omega\circ x\circ F, \text{ caso contrário.}
            \end{cases}
        \end{align*}
        \paragraph{}
            Com $\Omega_\alpha$ assim definido, podemos iterar partindo de $\emptyset$ 
            --- a função vazia --- obtendo uma sequência $\Omega_\alpha$ de funções.
            Elas vão ser, por construção, compatíveis, então consideramos a colagem:
            $$\Omega = \bigcup_{\alpha\in Ord}\Omega_\alpha$$
        \paragraph{}
            Fica claro que ela é de $\V{A}$ em $\V{B}$. Veremos mais a diante que 
            este mapa tem uma sorte de qualidades que dependem de quão decente é 
            nosso morfismo $\omega$. Ele se relaciona com a, agora familiar, 
            Hierarquia de Lévy, submodelos, entre outras coisas.
    \end{proposition}