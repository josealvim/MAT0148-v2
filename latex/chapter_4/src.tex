\chapter{Modelos a Valores Booleanos}
%    \epigraph{\justify
%            A successful attempt to express 
%            logical propositions by symbols, 
%            the laws of whose combinations 
%            should be founded upon the laws 
%            of the mental processes which 
%            they represent, would, so far, 
%            be a step towards a philosophical 
%            language.
%        }{\textit{George Boole -- The Mathematical Analysis of Logic. p.5}}    
    \cls
    \paragraph{}
        Queremos modelos diferentes de $\ZF$
        a fim de encontrar exemplos de modelos 
        com propriedades diferentes de $L$, 
        por exemplo, onde não valha o Axioma 
        da Escolha, ou a Hipótese do Contínuo.
    \paragraph{}
        Por questões cabalísticas, não podemos 
        de fato definir a verdade do $V$ de 
        maneira internalizada satisfatória
        \cite{Tarski}. E, por consequência,
        não é possível fazê-lo para $V^{(B)}$
        inteiro --- pois poderiamos trazer de 
        volta a definição para $V$. Nossa 
        abordagem é fazer a teoria de modelos 
        booleanos em submodelos base.
    \paragraph{}
        Ao longo deste capítulo, $B$ e $B'$ 
        serão álgebras Booleanas completas, 
        $B$ será munida das operações 
        $*$ para o \textit{meet}, $+$ para 
        o \textit{join}, $-$ para o 
        complemento e $\leq$ para a ordem 
        induzida. A implicação escreveremos 
        dupla $\RAR$.
    \begin{definition}{$ M^{(B)} $}
        Seja $B$ uma Álgebra Booleana Completa, 
        e seja $M\in V$ um conjunto transitivo 
        que modela $\ZF$.

        $$ \M{B}_{\alpha} = \SET{ f : \exists\beta\in M:\beta<\alpha\land\exists D\subseteq \M{B}_{\beta} : f\in D^B } $$
        $$ \M{B} = \SET{f : \exists\alpha: f\in \M{B}_{\alpha}} $$
    \end{definition}
    \paragraph{}
        Acima definimos um conjunto (pois a partir 
        de certo $\alpha$ não se adiciona mais 
        conjuntos) onde cada um de seus elementos 
        é uma função de um pedaço de um nível 
        anterior $B$-valorada.
    \paragraph{}
        A maneira de pensar sobre esta classe
        é que seus membros são representantes 
        de objetos ideais cujas propriedades 
        queremos capturar dentro da teoria de $\ZF$. 
        Cada representante decide 
        \textit{quanto} os objetos representados 
        por anteriores \textit{pertence}
        ao objeto que ele representa.
    \paragraph{}
        Trivialmente $\alpha<\beta\bim\M{B}_\alpha\subset\M{B}_\beta$
    \paragraph{}
        Já mencionamos que objetos em $\M{B}$ 
        representam objetos ideais, mas não 
        explicamos exatamente o que queremos 
        dizer com isso. É evidente que $\M{B}$
        não é transitivo, nem tem um conjunto 
        indutivo, logo não há esperança de 
        $\M{B}$ ser modelo transitivo de $\ZF$.
    \paragraph{}
        Mas, com uma relação não-\textit{standard}
        de pertinência e uma igualdade 
        adequadamente definida, é possível que 
        seja um modelo de $\ZF$. Como estamos 
        distorcendo a igualdade, os elementos de 
        $\M{B}$ não são os objetos do modelo, 
        estritamente falando, mas são coisas 
        habitando classes de equivalência que, 
        estas sim, são os verdadeiros objetos 
        do modelo.
    \paragraph{}
        Vamos, pois, definir uma ($B$) medida 
        de igualdade e de pertinência de maneira 
        recursiva: Primeiro considere a relação
    $$ \TUPLE{a}{b} \prec \TUPLE{\alpha}{\beta} \bim [a = \alpha \land b\in Dom(\beta)] \lor [a\in Dom(\alpha) \land b = \beta] $$
    \paragraph{}
        Queremos mostrar que, sobre 
        $\mathbb{M} = \M{B}\times\M{B}$, 
        esta relação é bem fundada --- 
        isto é, todo subconjunto de 
        $\mathbb{M}$ tem um elemento $\prec$-minimal 
        e $\prec$ é ``pequeno à esquerda''.
    \begin{lemma}{$\prec$ é Bem Fundada}
        Seja $B$ álgebra Booleana Completa, $M$ um 
        modelo interno transitivo de $\ZF$, então 
        $\prec$ é \textit{set-like} e minimalizável em 
        $\mathbb{M} = \M{B}\times\M{B}$.
        \begin{proof}
            \paragraph{}
                Primeiro lidemos com \textit{minimalizável},
                que significa dizer há um elemento minimal 
                em qualquer subconjunto:
            $$ \forall X \subseteq \mathbb{M}:\exists x_0 \in \mathbb{M}: \forall x\in X: \neg[x \prec x_0]$$
            \paragraph{}
                Como todo membro de $\M{B}$ surgiu neste conjunto em alguma etapa ordinal, seja $\rho_B$
                a função que associa a cada membro de $\M{B}$ o ordinal mínimo cuja etapa o contém. Dado 
                $\rho_B$ queremos extender esta função para uma útil em $\mathbb{M}$, no caso: 
                $\rho\TUPLE{a}{b}=\min\SET{\rho_B(a)}{\rho_B(b)}$.
            \paragraph{}
                Considere, pois, o conjunto de ordinais correspondente à imagem de $\mathbb{M}$ por $\rho$,
                este conjunto tem um mínimo $\gamma$ que é atingido por algum $\TUPLE{a}{b}\in\mathbb{M}$.
            \paragraph{}
                Seja, então, $\TUPLE{\alpha}{\beta}\in\mathbb{M}$
            \begin{prooftree}
                \AxiomC{$\scriptstyle[\alpha = a \land \beta \in Dom(b)]$ }
                \UnaryInfC{$\scriptstyle[\rho_B(\alpha) = \rho_B(a)] \land [\rho_B (\beta)< \rho_B(b)]$}
                \UnaryInfC{$\scriptstyle\rho \TUPLE{\alpha}{\beta} < \gamma$}
                
                \AxiomC{$\TUPLE{\alpha}{\beta} \prec \TUPLE{a}{b}$}
                \UnaryInfC{$\scriptstyle[\alpha = a \land \beta \in Dom(b)] \lor [\alpha \in Dom(a) \land \beta = b]$}
                
                \AxiomC{$\scriptstyle[\alpha \in Dom (a)\land\beta = b ]$ }
                \UnaryInfC{$\scriptstyle[\rho_B(\alpha) < \rho_B(a)] \land [\rho_B (\beta)= \rho_B(b)]$}
                \UnaryInfC{$\scriptstyle\rho \TUPLE{\alpha}{\beta} < \gamma$}

                \BinaryInfC{$\scriptstyle[\alpha = a \land \beta \in Dom(b)] \lor [\rho \TUPLE{\alpha}{\beta} < \gamma]$}
                \BinaryInfC{$\scriptstyle[\rho \TUPLE{\alpha}{\beta} < \gamma] \lor [\rho \TUPLE{\alpha}{\beta} < \gamma]$}
                \UnaryInfC{$\scriptstyle\rho \TUPLE{\alpha}{\beta} < \gamma$}
                \UnaryInfC{$\bot$}
            \end{prooftree}
            \paragraph{}
                Como se buscava, provamos que $\neg(\TUPLE{\alpha}{\beta} \prec \TUPLE{a}{b})$.\eop
            \paragraph{}
                Quanto a \textit{set-like}, que significa que a classe dos objetos menores que $x$ é um conjunto:
            $$\forall x\in \mathbb{M}:\exists X\subseteq\mathbb{M}: X = \SET{t : t \prec x}$$
            \paragraph{}
                É simples: $\mathbb{M}$ é conjunto, e se --- dado um $x$ dentro de $\mathbb{M}$ --- um $y$ é tal que 
                $y\prec x$ então, $y$ --- pela definição de $\prec$ deverá estar em $\mathbb{M}$ também.\eop
        \end{proof}
    \end{lemma}
    \paragraph{}
        Assim temos provado que esta relação auxiliar que definimos é bem fundada em qualquer $\M{B}$. Podemos 
        então, definir:
    \begin{align*}
        F(\langle{a},{b}\rangle, f) ={{\langle}}&\bigwedge_{x\in Dom(b)}[b(x)\RAR \pi_2\circ f(\TUPLE{x}{b})]*\bigwedge_{y\in Dom(a)}[a(y)\RAR \pi_3\circ f(\TUPLE{y}{a})],\\
            &\bigvee_{  x\in Dom(b)}[b(x)*    \pi_1\circ f(\TUPLE{a}{x})],\\
            &\bigvee_{  x\in Dom(a)}[a(x)*    \pi_1\circ f(\TUPLE{x}{b})]{\rangle}
    \end{align*}
    \paragraph{}
        Onde $\pi_i$ são as projeções usuais de \textit{tuplas}.
    \paragraph{}
        Pelo teorema da recursão transfinita, existe uma (única) $G$ função de domínio $\mathbb{M}=\M{B}\times\M{B}$ 
        tal que:
    $$\forall x\in\mathbb{M}:G(x) = F(x,G\restriction\SET{y: y \prec x})$$
    \begin{definition}{$B$-medidas de igualdade e pertinência}
        Para cada álgebra Booleana Completa $B$, teremos uma $G$ associada determinada pelo processo acima, definimos,
        então definimos $\Min{\_}{\_}$, que é a $B$ medida de pertiência em $\M{B}$ e $\Meq{\_}{\_}$, que é a medida de igualdade de 
        dois membros de $\M{B}$.
        $$ \Meq{a}{b} = \pi_1 \circ G\TUPLE{a}{b} = \left[\bigwedge_{x\in Dom(a)}a(x)\RAR (x\Min b)\right]*\left[\bigwedge_{y\in Dom(b)}b(y)\RAR (y\Min a)\right]$$
        $$ \Min{a}{b} = \pi_2 \circ G\TUPLE{a}{b} = \bigvee_{t\in Dom(b)} b(t)*(a\approx t)$$
    \end{definition}
    \paragraph{}
        De fato o que fizemos foi uma dupla-recursão transfinita: As coordenadas da tripla $G(x)$ representam 
        ``$\TUPLE{x_1 = x_2}{x_1 \in x_2}{x_1 \in x_2}$'' e precisamos destes valores para em níveis anteriores 
        para definir no atual nível.
    \begin{proposition}{Propriedades das medidas de $\M{B}$}
        \begin{enumerate}
            \item $ \Meq{x}{x} = 1$;
            \item $ \forall x\in Dom(y): y(x) \RAR \Min{x}{y} = 1$, no sentido da ordem $\leq$ de $B$;
            \item $ \Meq{x}{y} = \Meq{y}{x}$;
            \item $\Meq{x}{y} *  x(u)      \leq \Min{u}{y}$;
            \item $\Meq{x}{y} * \Meq{y}{z} \leq \Meq{x}{z}$;
            \item $\Meq{x}{y} * \Min{x}{z} \leq \Meq{y}{z}$;
            \item $\Meq{x}{y} * \Min{x}{z} \leq \Meq{y}{z}$;
        \end{enumerate}
        %%
        \begin{proof}

            ($\Meq{a}{a} = 1$):
            $$\Meq{a}{a} = \left[\bigwedge_{x\in Dom(a)} a(x)\RAR (\Min{x}{a})\right]*
                           \left[\bigwedge_{y\in Dom(a)} a(y)\RAR (\Min{y}{a})\right] = $$
            $$\Meq{a}{a} =       \bigwedge_{x\in Dom(a)} a(x)\RAR (\Min{x}{a})$$
            \paragraph{}
                Assumindo a hípotese indutiva, que $\forall x\in Dom(a):x\Meq x = 1$, então:
            $$\Min{x}{a} =       \bigvee_{  t\in Dom(a)} a(t)*    (\Meq{x}{t}) \geq a(x)*(\Meq{x}{x}) = a(x)$$
            \paragraph{}
                Substituindo acima temos:
            $$\Meq{a}{a} =       \bigwedge_{x\in Dom(a)} a(x)\RAR (\Min{x}{a}) = \bigwedge_{x\in Dom(a)}a(x)\RAR a(x) = 1$$\eop
        \end{proof}
        %%
        \begin{proof}
            
            ($ y(x) \leq \Min{x}{y}$):
            \paragraph{}
                Como $x\in Dom(y)$,
            $$ \Min{x}{y} = \bigvee_{t\in Dom(y)} y(t)*(\Meq{x}{t})\geq y(x)*(\Meq{x}{x}) = 1$$\eop
        \end{proof}
        %%
        \begin{proof}
            
            ($\Meq{a}{b} = \Meq{b}{a}$):
            \begin{align*}
                \Meq{a}{b}  = \left[\bigwedge_{x\in Dom(a)}a(x)\RAR (\Min{x}{b})\right]&*\left[\bigwedge_{y\in Dom(b)}b(y)\RAR (\Min{y}{a})\right] = \\
                            = \left[\bigwedge_{y\in Dom(b)}b(y)\RAR (\Min{y}{a})\right]&*\left[\bigwedge_{x\in Dom(a)}a(x)\RAR (\Min{x}{b})\right] = \Meq{b}{a}
            \end{align*}\eop
        \end{proof}
        %%
        \begin{proof}
            
            ($[\Meq{x}{y}] *  x(u) \leq [\Min{u}{y}]$):
            \begin{prooftree}
                \AxiomC{   $\Meq{x}{y} =  \bigwedge_{i, j}(x(i)\RAR[\Min{i}{y}])*(y(j)\RAR[\Min{j}{x}])$ }
                \UnaryInfC{$\Meq{x}{y} * x(u) \leq x(u) * (x(u)\RAR[\Min{u}{y}])*\bigwedge_{j}(y(j)\RAR[\Min{j}{x}])$}
                \UnaryInfC{$\Meq{x}{y} * x(u) \leq x(u) * (x(u)\RAR[\Min{u}{y}])$}
                \UnaryInfC{$\Meq{x}{y} * x(u) \leq \Min{u}{y}$}
            \end{prooftree}\eop
        \end{proof}
        %%
        \begin{proof}
            
            ($[\Meq{x}{y}] * [\Meq{y}{z}] \leq [\Meq{x}{z}]$):
            \paragraph{}
                Sejam $u,v,w\in\M{B}$. Vamos assumir, como passo indutivo, que
            $$ \forall x\in Dom(u):\forall y,z: (\Meq{x}{y}) * (\Meq{y}{z}) \leq (\Meq{x}{z})$$
            \paragraph{}
                Restringindo $y\in Dom(v), z\in Dom(w)$ temos:
            \begin{prooftree}
                \AxiomC{$(\Meq{x}{y}) * (\Meq{y}{z}) \leq (\Meq{x}{z})$}
                \UnaryInfC{$(\Meq{x}{y}) * (\Meq{y}{z}) * w(z) \leq (\Meq{x}{z}) * w(z)$}
                \UnaryInfC{$\bigvee_{z\in Dom(w)}(\Meq{x}{y}) * (\Meq{y}{z}) * w(z) \leq \bigvee_{z\in Dom(w)}(\Meq{x}{z}) * w(z)$}
                \UnaryInfC{$(\Meq{x}{y}) * (\Min{y}{w}) \leq (\Min{x}{w})$}
            \end{prooftree}
            \paragraph{}
                No entanto, temos, nas condições da hipótese indutiva, o seguinte resultado auxiliar:
            \begin{prooftree}
                \AxiomC{$a = x$, Hipótese Indutiva}
                \AxiomC{$    \Meq{a}{b}*a(d)\leq \Min{d}{b}$}
                \UnaryInfC{ $\Meq{a}{b}*\Meq{c}{d}*a(d)\leq\Min{d}{b}*\Meq{c}{d}$}
                \UnaryInfC{ $\Meq{a}{b}*\Meq{c}{d}*a(d)\leq\bigvee_{z} w(z)*\Meq{d}{z}*\Meq{c}{d}$}
                \BinaryInfC{$\Meq{a}{b}*\Meq{c}{d}*a(d)\leq\bigvee_{z} w(z)*\Meq{c}{z}$}
                \UnaryInfC{ $\Meq{a}{b}*\Meq{c}{d}*a(d)\leq\Min{c}{b}$}
                \UnaryInfC{ $\Meq{x}{y}*\Meq{u}{v}*x(u)\leq\Min{v}{y}$}
            \end{prooftree}
            \paragraph{}
                Tomando o supremo sobre $u$ no domínio de $x$, temos:
            $$ \Meq{x}{y}*\Min{v}{x}\leq \Min{v}{y}$$
            \paragraph{}
                Temos também que
            $$\Meq{x}{y} * x(u) \leq \Min{u}{y}$$
            \begin{prooftree}
                \AxiomC{$   \Meq{x}{y}*x(u)             \leq\Min{u}{y}$}
                \UnaryInfC{$\Meq{x}{y}*\Meq{y}{z}* x(u) \leq\Min{u}{y}*\Meq{y}{z}$}
                \UnaryInfC{$\Meq{x}{y}*\Meq{y}{z}* x(u) \leq\Min{u}{z}$}
                \UnaryInfC{$\Meq{x}{y}*\Meq{y}{z}       \leq   x(u)         \RAR\Min{u}{z}$}
                \UnaryInfC{$\Meq{x}{y}*\Meq{y}{z}       \leq\bigwedge_u x(u)\RAR\Min{u}{z}$}
            \end{prooftree}
            \paragraph{}
                Graças à hipótese indutiva, temos:
            $$\Meq{x}{y}*\Meq{y}{z} \leq\bigwedge_v z(v)\RAR \Min{v}{x}$$
            \paragraph{}
                E, logo, 
            $$\Meq{x}{y}*\Meq{y}{z}\leq \Meq{x}{z}$$\eop
        \end{proof}
    \end{proposition}
    \begin{definition}{Definição de $\hat{\_}$ e $\check{\_}$}
        $M$ sendo o nosso modelo base, então vamos considerar as seguintes operações:
        \begin{align*}
            \hat{\_}   :&M \longrightarrow \M{B}\\
                        &x \longmapsto   \hat{x} = \SET{\TUPLE{\hat{t}, 1}:t \in x}\\
                        & \\
            \check{\_} :&\M{2} \longrightarrow M\\
                        &y \longmapsto \check{y} = \SET{\check{t} : t\in y^{-1}(1)}
        \end{align*}
        \paragraph{}
            A definição formal é relativamente simples: fazemos a indução transfinita 
            sobre a relação bem fundada $\in$ no caso de $\hat{}$ e $\in Dom(x)$ no 
            caso de $\check{}$. Não há muito o que se verificar no que diz respeito 
            à validade da definição. Mas há um teorema importante sobre estes operadores.
    \end{definition}
    \begin{theorem}{Caracterização das verdades atômicas de $\M{2}$}
        \begin{enumerate}[label=(\alph*)]
            \item $\forall x\in M   :\check{\hat{x}} = x$
            \item $\forall y\in\M{2}:\Meq{\hat{\check{y}}}{y} = 1$
            \item $\forall x_1, x_2 \in M   : x_1 =  x_2 \bim \Meq{\hat x_1}{\hat x_2} = 1$
            \item $\forall x_1, x_2 \in M   : x_1\in x_2 \bim \Min{\hat x_1}{\hat x_2} = 1$
            \item $\forall y_1, y_2 \in\M{2}:\Meq{y_1}{y_2}=1\bim\check{y_1} = \check{y_2}$
            \item $\forall y_1, y_2 \in\M{2}:\Min{y_1}{y_2}=1\bim\check{y_1}\in\check{y_2}$
        \end{enumerate}
        \paragraph{}
            Isto será muito útil para mostrar que $\M{2}$ é essencialmente $M$.
        \begin{proof}
            $ $\\$(a)$\\
            Seja $x\in M$,
            $\check{y} = \SET{\check{t} : t\in y^{-1}(1)}$, mas então, 
            $\check{\hat{x}} = \SET{\check{t} : t\in \hat{x}^{-1}(1)}$, que por sua vez
            $\check{\hat{x}} = \SET{\check{t} : t\in Dom(\hat{x})}$. Suponha que para 
            todo $m$ de rank menor que $\rho(x)$ valha o que queremos provar. Então 
            $\check{\hat{x}} = \SET{\check{\hat{m}} : m\in x}$, mas pela hipótese 
            indutiva isso é justamente $x$. Então a etapa sucessora está provada. No 
            caso em que o rank de $x$ é um limite, o argumento é práticamente o mesmo.
        \end{proof}
        \begin{proof}
            $ $\\$(b)$\\
            Seja $y\in\M{2}$, que, para hipótese indutiva, para cada um de seus membros 
            valha nossa asserção. 
        \end{proof}
    \end{theorem} 